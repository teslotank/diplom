\begin{center}
ОПРЕДЕЛЕНИЯ, ОБОЗНАЧЕНИЯ И СОКРАЩЕНИЯ
\end{center}

В настоящей работе применяют следующие термины с соответствующими определениями:
СЭМ -- сканирующий электронный микроскоп\\
МНП -- метод непрерывных потерь\\
ПММА -- полиметилметакрилат\\
$e$ -- заряд электрона\\
$k$ -- постоянная Больцмана\\
$d_c$ -- диаметр кроссовера\\
$d_p$ -- диаметр конечного пятна\\
$D^0$ -- чувствительность поглощения\\
$T^0$ -- чувствительность экспонирования\\
$\eta$ -- вязкость проявителя\\
$Q$ -- коэффициент диффузии проявителя\\
$\Sigma$ -- толщина диффузионного слоя\\
$\gamma$ -- контрастность\\
$v$ -- скорость проявления\\
$N_f$~-- плотность потока электронов, сохраняющих направление\\
$N_d$~-- плотность потока диффундирующих электронов\\
$W_{tr}$~-- транспортная частота столкновений\\
$\bar{\varepsilon}$~-- потеря энергии на единицу пробега\\
$W_{sa,tr}$~-- малоугловой компонент частоты столкновений\\
$W_d$ -- плотность поглощенной энергии диффузного компонента
