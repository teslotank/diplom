\chapter{Экспериментальная часть}
Используя расчетную программу построенную на базе приведенной выше математической модели, было проведено моделирование нормального падения моноэнергетического пучка электронов на подложку с нанесенным на нее слоем резиста. Энергия электронов задана на уровне 20 кэВ, подложка выполнена из кремния толщиной 5000 нм, а резист выполнен из ПММА и имел толщину 500 нм.

\begin{figure}[h]
    \center
    \includegraphics[width=.47\textwidth]{example}
    \caption{Схематическое изображение подложки с нанесенным на нее резистом}
    \label{fig:example}
\end{figure}
В налетающем пучке распределение плотности энергии имеет вид, представленный на рисунке~\ref{fig:beam}.
\begin{figure}[h]
    \center
    \includegraphics[width=.95\textwidth]{beam}
    \caption{радиальное распределение плотности энергии в налетающем пучке}
    \label{fig:beam}
\end{figure}

Как можно видеть из графика, представленного на рис.~\ref{fig:fluxes}, при данном значении энергии лишь малая часть электронов отражается обратно, в основном они диффундируют. Что говорит о малом количестве электронов участвующих в эффекте близости.
% здесь можно приплести про эффект близости
\begin{figure}[h]
    \center
    \includegraphics[width=.95\textwidth]{fluxes}
    \caption{Доля потоков электронов:
    1~--~электроны, перешедшие из нерассеянных в диффузионные;
2~--~диффундировавшие электроны с энергией $E$;
3~--~отраженные электроны с энергией от $E_0$ до $E$;
4~--~прошедшие насквозь через подложку электроны с энергией от $E_0$ до $E$.
}
    \label{fig:fluxes}
\end{figure}

По мере прохождения доля диффузионных электронов растет.
\begin{figure}[h]
    \center
    \includegraphics[width=.95\textwidth]{rates}
    \caption{Соотношение диффузионных и нерассеянных электронов: 1~--~отношение потока нерассеянных электронов к потоку диффузионных;
2~--~доля потока нерассеянных электронов;
3~--~доля потока диффузионных электронов.
}
    \label{fig:rates}
\end{figure}
При прохождении в глубь резиста, а впоследствии и подложки, пучок уширяется
\begin{figure}[h]
    \center
    \includegraphics[width=.95\textwidth]{sequence}
    \caption{Поперечные размеры пучка:
1~--~полуширина пучка нерассеянных электронов;
2~--~полуширина пучка диффузионных электронов по 1-му гауссиану;
3~--~полуширина пучка диффузионных электронов по 2-му гауссиану;
4~--~полуширина пучка диффузионных электронов по 3-му гауссиану;
5~--~полуширина пучка диффузионных электронов по 4-му гауссиану.}
    \label{fig:sequence}
\end{figure}

Из зависимости показанной на рис.~\ref{fig:sequence} видно что пр прохождении пучка через резист на глубине примерно 230 нм находится максимум поглощения энергии пучка резистом и мишенью в целом. Следующий максимум поглощения находится в приграничной зоне подложки и резиста, далее интенсивность спадает. Данные результаты можно интерпретировать следующим образом. Пи использовании относительно низкой энергии электронного пучка при прохождении его через резист он начинает диффундировать, и в последствии поглощаться резистом, так как энергии электронов относительно не велики. При переходе в подложку, те электроны которые прошли через резист 2 на рис.~\ref{fig:fluxes} и не были поглощены но потеряли энергию в результате диффузии, при переходе в более плотную среду, поглощаются.
\begin{figure}[h]
    \center
    \includegraphics[width=.95\textwidth]{absorption}
    \caption{Зависимость от глубины плотности поглощенной энергии нерассеянных электронов в расчете на 1 электрон 1 для пучка радиусом 24.3 нм}
    \label{fig:absorption}
\end{figure}

%можно сделать вывод что , при пр
