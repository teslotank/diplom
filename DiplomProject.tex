     \documentclass[a4paper,12pt]{report} %размер бумаги устанавливаем А4, шрифт 12пунктов
    \usepackage[T2A]{fontenc}
    \usepackage[utf8x]{inputenc}%включаем свою кодировку: koi8-r или utf8 в UNIX, cp1251 в Windows
    \usepackage[english,russian]{babel}%используем русский и английский языки с переносами
    \usepackage{amssymb,amsfonts,amsmath,mathtext,cite,enumerate,float} %подключаем нужные пакеты расширений
    \usepackage[dvips]{graphicx} %хотим вставлять в диплом рисунки?
    \graphicspath{{images/}}%путь к рисункам

    \makeatletter
    \renewcommand{\@biblabel}[1]{#1.} % Заменяем библиографию с квадратных скобок на точку:
    \makeatother

    \usepackage{geometry} % Меняем поля страницы
    \geometry{left=2cm}% левое поле
    \geometry{right=1.5cm}% правое поле
    \geometry{top=1cm}% верхнее поле
    \geometry{bottom=2cm}% нижнее поле

    \renewcommand{\theenumi}{\arabic{enumi}}% Меняем везде перечисления на цифра.цифра
    \renewcommand{\labelenumi}{\arabic{enumi}}% Меняем везде перечисления на цифра.цифра
    \renewcommand{\theenumii}{.\arabic{enumii}}% Меняем везде перечисления на цифра.цифра
    \renewcommand{\labelenumii}{\arabic{enumi}.\arabic{enumii}.}% Меняем везде перечисления на цифра.цифра
    \renewcommand{\theenumiii}{.\arabic{enumiii}}% Меняем везде перечисления на цифра.цифра
    \renewcommand{\labelenumiii}{\arabic{enumi}.\arabic{enumii}.\arabic{enumiii}.}% Меняем везде перечисления на цифра.цифра

    \begin{document}
         \begin{titlepage}
    \newpage

    \begin{center}
    ФЕДЕРАЛЬНОЕ АГЕНТСТВО ПО ОБРАЗОВАНИЮ РФ \\
    \vspace{1cm}
    Н-СКИЙ АРБУЗО-ЛИТЕЙНЫЙ ИНСТИТУТ \\*
    (ГОСУДАРСТВЕННЫЙ УНИВЕРСИТЕТ) \\*
    \hrulefill
    \end{center}

    \flushright{КАФЕДРА \No ХХХ}

    \vspace{8em}

    \begin{center}
    \Large Пояснительная записка \\ к дипломному проекту на тему:
    \end{center}

    \vspace{2.5em}

    \begin{center}
    \textsc{\textbf{исследование торсионных наногенераторов \linebreak стволовых клеток для борьбы с терроризмом}}
    \end{center}

    \vspace{6em}

    \begin{flushleft}
    Студент--дипломник \hrulefill Пупкин А.А. \\
    \vspace{1.5em}
    Научный руководитель \\
    доцент \hrulefill Иванов Б.Б.\\
    \vspace{1.5em}
    Рецензент \\
    к.ф.-м.н., в.н.с. АБВГ \hrulefill Петров В.В.\\
    \vspace{1.5em}
    Зав. кафедрой \No ХХХ \\
    д.ф-м.н, профессор \hrulefill Сидоров Г.Г.
    \end{flushleft}

    \vspace{\fill}

    \begin{center}
    Н-ск 2000
    \end{center}

    \end{titlepage}% это титульный лист
    
\begin{titlepage}

\begin{center}
\normalsize
МИНИСТЕРСТВО ОБРАЗОВАНИЯ И НАУКИ РОССИЙСКОЙ ФЕДЕРАЦИИ\\
ВОЛГОГРАДСКИЙ ГОСУДАРСТВЕННЫЙ ТЕХНИЧЕСКИЙ УНИВЕРСИТЕТ\\
ФАКУЛЬТЕТ ЭЛЕКТРОНИКИ И ВЫЧИСЛИТЕЛЬНОЙ ТЕХНИКИ\\
КАФЕДРА ФИЗИКИ\\[5cm]
\Large
\bf
КУРСОВАЯ РАБОТА \\ ПО ОСНОВАМ МОДЕЛИРОВАНИЯ\\
ВЗАИМОДЕЙСТВИЕ И ПРОХОЖДЕНИЕ\\ ЭЛЕКТРОНОВ ЧЕРЕЗ ВЕЩЕСТВО\\[3cm]
\end{center}

\hfill\parbox{6cm}{
\leftline{Выполнил студент гр.Ф-469}\\
\leftline{Байдаченко В.А.}\\
\leftline{Проверил:}\\
\leftline{проф., к.ф.-м.н. Смоляр В.А.}\\
\leftline{Дата:             }\\
\leftline{Оценка:           }\\
\leftline{Подпись:          }\\}
\begin{center}
\\[2.75cm]
Волгоград, 2013 г.\\
\end{center}

\end{titlepage}
 


    \tableofcontents % это оглавление, которое генерируется автоматически
    
\begin{titlepage}

\chapter*{Введение}
\addcontentsline{toc}{chapter}{Введение}

Миниатюризация элементов интегральных схем есть способ увеличения их
производительности и эффективности. Поэтому существует потребность как
в развитии наноструктурирования, так и создания специализированных
структур на чипах с элементами, имеющими нанометровые размеры в таких
областях как: оптоэлектроника, рентгеновская оптика, исследования в
области физики низких температур и квантово-размерных эффектов, новых
материалов, таких как двумерный графен и т.д.
Структурирование с помощью электронной литографии является самым
удобным методом создания объектов ввиду своей гибкости и оперативности.


\end{titlepage}
 

    
\begin{titlepage}

\begin{center}

\chapter{условия от которых зависит разрешение литоргафии}%\label{chapter_Inelastic}}
мамку товю ебал демат ты проклятый
\section{ебаная}
мамку товю ебал демат ты проклятый
\subsection{лурком}
мамку товю ебал демат ты проклятый
\end{center}

\end{titlepage}
 


    
    \end{document}