\chapter*{Заключение}
\addcontentsline{toc}{chapter}{Заключение}
В результате использования низкого (около 20 кВ) напряжения у нас получается золотая середина,
так как при использовании более низкого напряжения (\(\sim\)10 кВ) у нас получается что сильная диффузия электронов начинается прямо в слое пмма и сильно уменьшает качество маски при малых размерах ячеек.
при использовании высокого напряжения сильно проявляется эффект близости так же уменьшающий качество маски
в нашем же случае, по результатам расчета программы видно что доля обратно рассеянных электронов весьма мала, а поглощение начинается сразу же при вхождении в слой подложки в виду того что не велика энергия электронов, и влияние эффекта близости в таком случае весьма мало.
