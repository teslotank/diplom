\chapter*{Заключение}
\addcontentsline{toc}{chapter}{Заключение}
В результате использования низкого (около 20 кВ) напряжения у нас получается остров стабильности,
так как при использовании более низкого напряжения (\(\sim\)10 кВ) у нас получается что сильная диффузия электронов начинается прямо в слое пмма и сильно уменьшает качество маски при малых размерах ячеек.
при использовании высокого напряжения сильно проявляется эффект близости так же уменьшающий качество маски
в нашем же случае, по результатам расчета программы видно что доля обратно рассеянных электронов весьма мала, а поглощение начинается сразу же при вхождении в слой подложки в виду того что не велика энергия электронов, и влияние эффекта близости в таком случае весьма мало.


При изготовлении малых партий (большой номенклатуры) заказных логических схем из базового кристалла прямое рисование электронным лучем экономичнее, чем фотопечать через шаблон. Благодаря высокой разрешающей способности ЭЛ-литография будет и дальше использоваться при изготовлении шаблонов для световых, рентгеновских и ионных пучков. Кроме того, точность совмещения на каждом кристалле при ЭЛ-экспонировании составляет ±0.1 мкм, что является решающим преимуществом перед всеми остальными видами экспонирования.
