\chapter*{Заключение}
\addcontentsline{toc}{chapter}{Заключение}
В результате использования низкого (около 20 кВ) напряжения у нас получается остров стабильности,
так как при использовании более низкого напряжения (\(\sim\)10 кВ) у нас получается что сильная диффузия электронов начинается прямо в слое резиста и сильно уменьшает качество маски при малых размерах структуры.
При использовании высокого напряжения (\(\sim\)100 кВ), сильно проявляется эффект близости так же уменьшающий качество маски.

По результатам расчета программы видно, что доля обратно рассеянных электронов весьма мала, а поглощение начинается на средней глубине резиста, что говорит о высоком кпд данного метода экспонирования. Электроны с малой энергией поглощаются при вхождении в слой подложки ввиду потерь энергии при прохождении через резист, и их остаточная энергия сравнительно мала, поэтому они легко поглощаются подложкой. В этом случае получается, что при использовании низкоэнергетических пучков электронов экспонирование ведется преимущественно вперед рассеянными электронами и размытие изображения минимально. А эффект близости проявляется весьма слабо,так как доля обратно рассеянных электронов сравнительно мала.

%При изготовлении малых партий (большой номенклатуры) заказных логических схем из базового кристалла прямое рисование электронным лучем %экономичнее, чем фотопечать через шаблон. Благодаря высокой разрешающей способности ЭЛ-литография будет и дальше использоваться при изготовлении шаблонов для световых, рентгеновских и ионных пучков. Кроме того, точность совмещения на каждом кристалле при ЭЛ-экспонировании составляет ±0.1 мкм, что является решающим преимуществом перед всеми остальными видами экспонирования.
