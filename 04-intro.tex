\chapter*{ВВЕДЕНИЕ}
\addcontentsline{toc}{chapter}{Введение}
\vspace*{1cm}

Миниатюризация элементов интегральных схем есть способ увеличения их
производительности и эффективности. Поэтому существует потребность как
в развитии наноструктурирования, так и создания специализированных
структур на чипах с элементами, имеющими нанометровые размеры в таких
областях как: оптоэлектроника, рентгеновская оптика, исследования в
области физики низких температур и квантово-размерных эффектов, новых
материалов, таких как двумерный графен и т.д.
Структурирование с помощью электронной литографии является самым
удобным методом создания объектов ввиду своей гибкости и оперативности.
Так как электронная литография обладает большим потенциалом получения 
мкроструктур с высоким разрешением, доступными источниками излучения и
большой глубиной фокуса. Однако в любом методе есть как положительные 
так и негативные стороны. Для электронной литографйии этой стороной 
является эффекты происходящие в подложке резиста. Эффекты близости
наиболее существенно влияют на качество и характер получаемой
микроструктуры. Эффекты близости проявляются как искажение получаемого 
на подложке изображения вследствие упругого и неупругого рассеяния 
электронов на подложке. Электроны, рассеянные на атомах подложки, 
проникают в прилежащие к лучу области резиста, производя его 
дополнительное экспонирование, вызывая тем самым размытие изображения.
В зависимости от отсутствия или наличия ближайших “соседей” наблюдается 
соответственно внутренний или взаимный эффект близости. Внутренний эффект 
близости, обусловленный обратным рассеянием электронов за пределы 
непосредственно экспонируемой области, приводит к тому, что уединенные 
мелкие элементы топологии приходится экспонировать с дозой $Q$, заметно 
большей $Q_0$, необходимой для больших фигур. 
Одним из методов позволяющих решить проблему эффектов близости и
его последствия отражающиеся на качестве полученной микроструктуы, 
является использование электронов низких энергий. Перспективным 
направлением в этой области является электроны с энергией (\(\sim\)20) кэВ.









