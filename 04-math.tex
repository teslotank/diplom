Задачи о транспорте быстрых электронов в многослойных мишенях в настоящее
время решаются почти исключительно методом Монте-Карло, феноменологические
модели здесь практически непригодны, а аналитические методы решения кинетического
уравнения недостаточно разработаны даже для однородных слоев и ограничиваются
частными задачами — это либо задача об обратном рассеянии, либо, что значительно
реже, задача об энерговыделении для широкого пучка, падающего на однородную
мишень. Известным нам исключением является работа [1], в которой двухгрупповая
модель расщепления кинетического уравнения Спенсера была применена для расчета
энерговыделения широкого пучка в двухслойной мишени. Однако в этой работе
уравнение Спенсера записано для плотности потока электронов в фазовом пространстве,
включающем остаточный пробег электронов, и поэтому оно неприменимо для
неоднородных сред, так как в таких средах нет однозначного соответствия между
энергией электронов и их остаточным пробегом и пробег не имеет физического смысла.
Для многослойных мишеней следует применять кинетическое уравнение для электронов в
фазовом пространстве. Включающем энергию, как это сделано в работе [2].
В наших предыдущих работах [3–5] было описано расщепление транспортного
уравнения переноса электронов на основе соединения в единой математической модели
двух предельных приближений — малоуглового и транспортного. В этой модели частота
упругих столкновений аппроксимируется суммой изотропного и малоуглового
компонентов, а плотность потока электронов представляется в виде суммы двух
компонентов — сохраняющего первоначальное направление и диффундирующего.
В случае нормального падения на образец, состоящий из n слоев, точечного пучка
электронов с начальной энергией E0 не происходит слишком большого усложнения
задачи, и она остается в пределах возможностей применяемых аналитических и
численных методов. Плотность потока электронов в среде также представляется в виде
суммы
(1)
\begin{equation}
N(z,\vec{r},\vec{\Omega},E)=N_f (z,\vec{r},\vec{\Omega},E)+N_d (z,\vec{r},\vec{\Omega},E),
\label{eq:}
\end{equation}
где $N_f$ и $N_d$ – плотности потока электронов, сохраняющих направление и
диффундирующих, ось $z$ с началом в точке падения пучка направлена в глубину мишени,
$\vec{r}$ – вектор, ортогональный оси $z$, \vec{\Omega} – направление скорости и $E$ – энергия электронов.
Для функции $N_f$ было получено аналитическое решение в однородном слое [5] и в
многослойной структуре [2]. Здесь мы запишем аналитическое решение для $N_f$ в
многослойной мишени в несколько иной форме. Транспортная частота и средние потери
энергии в такой мишени являются ступенчатыми функциями относительно переменной $z$
и могут быть представлены с помощью характеристических функций \chi_i(E) i-того слоя –
произведения единичных ступенчатых функций Хевисайда $U(x)$


\begin{equation}
\bar{\varepsilon}(E,z)=\sum_{i=1}^n \chi_i (E)\bar{\varepsilon}^{(i)} (E),
\label{eq:2}
\end{equation}

\begin{equation}
W_tr(E,z)=\sum_{i=1}^n \chi_i (E)W_{tr}^{(i)} (E),
\label{eq:3}
\end{equation}

\begin{equation}
W_{sa,tr}(E,z)=\sum_{i=1}^n \chi_i (E)W_{sa,tr}^{(i)} (E),
\label{eq:4}
\end{equation}
где
\chi_i(E)=U(E_{i-1}-E)U(E-E_i),
$Ei$ – энергия прямоидущих электронов на правой границе i-того слоя, $W_tr(E,z)$ –
транспортная частота столкновений, $\bar{\varepsilon}(E,z)$ – потеря энергии на единицу пробега, W_{sa,tr}(E,z) – малоугловой компонент частоты столкновений.
Плотность малоуглового компонента потока имеет вид [2]

\begin{equation}
N_f(z,\vec{r},\vec{\Theta},E)=\frac{\delta[z-s(E,z)]}{[\pi \bar{\varepsilon}(E,z)\sigma^2(z)]}\times \exp \frac{[\frac{(-\vec{\Theta}^2 C_2(z)-\vec{r}^2C_0 (z)+2(\vec{r}\vec{\Theta})C_1(z))}{\sigma^2(z)}]}{[\pi\sigma^2(z)]} \times \exp\left(-\int_{E}^{E_0} \frac{w_{tr}(E',z)}{\bar{\varepsilon}(E',z)}dE'\right),
\label{eq:6}
\end{equation}

где



Функция s(E, z) имеет смысл среднего пути электрона, сохраняющего первоначальное
направление вдоль оси z, когда его энергия уменьшается от значения E0 в точке z=0 до E
на глубине z.
Плотность потока диффузного компонента N d ( z, r , Ω, E ) является решением
уравнения [2]
∂
[ε ( E , z )N d ( z, r , Ω, E )] + Ω∇N d ( z, r , Ω, E ) =
−
∂E
1
dΩ′wtr ( E , z )[N d ( z, Ω′, E ) − N d ( z, r Ω, E )] + .
(10)
=
4π ∫
1
wtr ( E , z )∫ dΩ′N f ( z, r ′, Ω′, E )
+
4π
Используя аналитическое решение (6) и численное решение уравнения (10), мы
можем определить любые дифференциальные и интегральные характеристики переноса
электронов в мишени. Расчеты распределений обратно рассеянных и прошедших
электронов обсуждались ранее [2]. В настоящей работе мы рассмотрим распределение
плотности поглощенной энергии электронов.
Учитывая аксиальную симметрию задачи, представим плотность поглощенной
энергии в виде
E0
(11)
W ( z, r ) = ∫ dE ε ( E , z )∫ dΩN ( z, r , ΩE ) = W f ( z, r ) + W d ( z, r ) ,
0
где
E0 0 
(13)
1 −1 E0 
(12)
1 E0
−1 0
W f ( z, r ) = ∫ dE ε ( E , z )2π ∫ d (cosθ )N f ( z, r,θ , E ) ,
W d ( z, r ) = ∫ dE ε ( E , z ) 2π ∫ d (cosθ )N d ( z, r,θ , E ) = ∫ dE ε ( E , z )N d 0 ( z, r, E ) ,
0
1
(14)
N d 0 ( z, r, E ) = 2π ∫ d (cosθ )N d ( z, r,θ , E ) .
−1
Здесь использована цилиндрическая система координат,
r = r
— радиальное
расстояние, θ — угол между осью z и вектором Ω .
Используя (6) и выполняя интегрирование в (12), получаем гауссиан
W f ( z, r ) = A f ( z )exp r 2 σ 2 ( z ) πσ 2 ( z ) ,
(15)
f
f
(
)(
)
с полушириной σ f ( z ) = C 2 ( z ) и амплитудой
E0
 E0

A f ( z ) = ∫ dE δ ( z − s( E , z )) exp ∫ dE ′ wtr ( E ′, z ) ε ( E ′, z )  .


0
0

Используя представление двумерной δ- функции с помощью последовательности
непрерывных функций [7]
δ ( r ) = lim exp − r 2 / σ 2 πσ 2
(17)
(16)
σ →0
(
)( )
из (16) получаем
(18)
limW f ( z, r ) = A f ( 0 )δ ( r ) ,
z →0
так что распределение W f ( z, r ) является сингулярным вблизи точки падения
электронного пучка. Отметим, что
A f ( 0 ) = ε ( E = 0, z = 0 ) .
(19)
Плотность поглощенной энергии диффузного компонента Wd(z, r) может быть
рассчитана с помощью функции Nd0(z, r, E), которая является решением краевой задачи.
Введя для удобства новую переменную τ=E0-E – остаточную энергию электронов,
представим задачу для функции Nd0(z, r, τ) в стандартном виде [12]
∂N d 0 ∂τ = ∂ ∂z [A (τ , z )∂N d 0 ∂z ] +
(20)
+ (1 r ) ∂ ∂r [rA (τ , z )∂N d 0 ∂r ] + B (τ , z )∂N d 0 ∂z − G (τ , z )N d 0 + ,
+ (exp( − r 2 C 2 ( z ) )πC 2 ( z ))( g(τ , z ) ε (τ , z ) )δ ( z − s(τ , z ))
где
(21)
(22)
(23)
A (τ , z ) = 1 (ε (τ , z )wtr (τ , z )) ,
B (τ , z ) = (1 − wtr (τ , z ) )∂(1 ε (τ , z ) ) ∂z ,
G (τ , z ) = ∂ ln( ε (τ , z )) ∂τ ,
E0
(24)
g(τ , z ) = [wtr (τ , z ) ε (τ , z )]exp( − ∫ dτ ′wtr (τ ′, z ) ε (τ ′, z ) ) .
0
Для функций τ (τ , z ), wtr (τ , z ) и s(τ,z) используем аппроксимации, описанные в [8].
Пусть ξ1,...,ξn – точки разрыва функции ε (τ , z ) по отношению к переменной z (эти
точки являются границами слоев). Тогда
n
(25)
{ }
[
]
B (τ , z ) = −( wtr (τ , z ) )∑ ε −1 (τ ,ξ i ) δ ( z − ξ i ) ,
1
i =1
где ε −1 обозначает скачек
{
}
ε −1 (τ ,ξ i ) = ε −1 (τ ,ξ i + 0 ) − ε −1 (τ ,ξ i − 0 ) .
(26)
В однослойной мишени имеем B(τ,z)=0.
Уравнение (20) должно быть решено в области
(27)
( z, r,τ ) ∈ [0, z макс ]× [0, rмакс ]× [0, E 0 ] ,
где величина zмакс определяется структурой мишени, а значение rмакс должно быть
конечным из вычислительных соображений. Представляется естественным выбрать rмакс
много большим, чем максимальная оценка длины свободного пробега в слоях.
Граничные и условия Маршака в стандартной форме [6] имеют вид
[0,5 N d 0 ε (τ , z ) − A (τ , z ) ∂N do ∂z ]z =0 = 0 ,
(28)
(29)
(30)
(31)
[0,5 N d 0 ε (τ , z ) + A (τ , z ) ∂N do
[0,5 N d 0 ε (τ , z ) + A (τ , z ) ∂N do
[∂N d 0 ∂r ]r=0 = 0 .
∂z ]z =zмакс = 0 ,
∂r ]r=rмакс = 0 ,
Кроме того,
N d 0 ( z, r,τ = 0 ) = 0 .

