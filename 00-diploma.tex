\documentclass[a4paper,14pt]{extreport} %размер бумаги устанавливаем А4, шрифт 12пунктов
\usepackage[T2A]{fontenc}
\usepackage[utf8x]{inputenc}%включаем свою кодировку: koi8-r или utf8 в UNIX, cp1251 в Windows
\usepackage[english,russian]{babel}%используем русский и английский языки с переносами
\usepackage{indentfirst}
\usepackage{amssymb,amsfonts,amsmath,mathtext,cite,enumerate,float} %подключаем нужные пакеты расширений
\usepackage{graphicx} %хотим вставлять в диплом рисунки?
\graphicspath{{images/},{plots/}}%путь к рисункам

\usepackage{multicol, multirow, array}
\newcolumntype{C}[1]{>{\centering\arraybackslash}m{#1\textwidth}}
\renewcommand{\arraystretch}{1.2}
\usepackage{times}
\renewcommand{\rmdefault}{ftm}

\usepackage{geometry} % Меняем поля страницы
\geometry{left=2cm}% левое поле
\geometry{right=1.5cm}% правое поле
\geometry{top=1cm}% верхнее поле
\geometry{bottom=2cm}% нижнее поле

\usepackage{tocloft}
\renewcommand{\cfttoctitlefont}{\normalfont}
\makeatletter 
\renewcommand*\l@chapter[2]{% 
  \ifnum \c@tocdepth >\z@ 
    \addpenalty\@secpenalty 
    \addvspace{1.0em \@plus\p@}% 
    \setlength\@tempdima{1.5em}% 
    \begingroup 
      \parindent \z@ \rightskip \@pnumwidth 
      \parfillskip -\@pnumwidth 
      \leavevmode 
      \advance\leftskip\@tempdima 
      \hskip -\leftskip 
      #1\nobreak\cftdotfill{\cftdotsep} \nobreak\hb@xt@\@pnumwidth{\hss #2}\par 
    \endgroup 
  \fi} 
\renewcommand*\l@section[2]{% 
  \ifnum \c@tocdepth >\z@ 
    \addpenalty\@secpenalty 
    \addvspace{1.0em \@plus\p@}% 
    \setlength\@tempdima{1.5em}% 
    \begingroup 
      \parindent \z@ \rightskip \@pnumwidth 
      \parfillskip -\@pnumwidth 
      \leavevmode 
      \advance\leftskip\@tempdima 
      \hskip -\leftskip 
      #1\nobreak\cftdotfill{\cftdotsep} \nobreak\hb@xt@\@pnumwidth{\hss #2}\par 
    \endgroup 
  \fi} 
\makeatother
\usepackage{setspace}

\usepackage{titlesec}
\usepackage[square, numbers, sort&compress]{natbib}
\makeatletter
% \bibliographystyle{srt}
\renewcommand{\@biblabel}[1]{#1} 
\makeatother
\titleformat{\chapter}
    {\normalsize}
    {\thechapter}
    {1em}{}

\titleformat{\section}
    {\normalsize}
    {\thesection}
    {1em}{}

\titleformat{\subsection}
    {\normalsize}
    {\thesubsection}
    {1em}{}
\renewcommand{\bibsection}{%
    \titleformat{\chapter} {\normalfont\filcenter}{}{}{}
    \titlespacing*{\chapter}{\parindent}{*4}{*0}
    \chapter*{ \bibname \markboth{\bibname}{\bibname}}}
% Настройка вертикальных и горизонтальных отступов
\titlespacing*{\chapter}{\parindent}{*4}{*0}
\titlespacing*{\section}{\parindent}{*4}{*0}
\titlespacing*{\subsection}{\parindent}{*4}{*0}

\renewcommand{\theenumi}{\arabic{enumi}}
\renewcommand{\labelenumi}{\arabic{enumi})}
\renewcommand{\theenumii}{.\arabic{enumii}}
\renewcommand{\labelenumii}{\arabic{enumi}.\arabic{enumii})}
\renewcommand{\theenumiii}{.\arabic{enumiii}}
\renewcommand{\labelenumiii}{\arabic{enumi}.\arabic{enumii}.\arabic{enumiii})}

\newcolumntype{C}[1]{>{\centering\arraybackslash}m{#1\textwidth}}
\renewcommand{\arraystretch}{1.2}

\renewcommand{\small}{\fontsize{12pt}{14.4pt}\selectfont}
\usepackage{caption}
\DeclareCaptionLabelFormat{figure}{Рисунок #2}
\DeclareCaptionLabelFormat{table}{Таблица #2}
\DeclareCaptionLabelSeparator{sep}{~---~}
\captionsetup{labelsep=sep,justification=centering,font=small}
\captionsetup[figure]{labelformat=figure}
\captionsetup[table]{labelformat=table}

\begin{document}
\onehalfspacing
\setlength{\parindent}{1.25cm}
\begin{titlepage}
    \begin{center}
    Министерство образования и науки РФ \\
    \vspace{.5cm}
    ГОСУДАРСТВЕННОЕ ОБРАЗОВАТЕЛЬНОЕ УЧРЕЖДЖЕНИЕ\\*
    ВЫСШЕГО ПРОФЕССИОНАЛЬНОГО ОБРАЗОВАНИЯ\\*
    <<ВОЛГОГРАДСКИЙ ГОСУДАРСТВЕННЫЙ ТЕХНИЧЕСКИЙ УНИВЕРСИТЕТ>>\\
    \vspace{.5cm}
    Кафедра <<ФИЗИКА>>
    \vspace{.5cm}
    \end{center}
    \begin{flushright}
        УТВЕРЖДАЮ\\
        Заведующий кафедрой <<Физика>>\\
        проф., д-р физ.-мат. наук А.~Г.~Шеин\\
        \vspace{1cm}
        <<\underline{\ \ \ \ }>>\underline{\ \ \ \ \ \ \ \ \ \ \ \ \ \ } 2013г.
    \end{flushright}
    
    \vspace{.5em}

    \begin{center}
    \LARGE \textbf{ПОЯСНИТЕЛЬНАЯ ЗАПИСКА} \\
    \large К ВЫПУСКНОЙ РАБОТЕ
    \end{center}

    \vspace{.5em}

    \begin{center}
    на тему \underline{низкоэнергетическая электронная литография}
    \end{center}

    \vspace{.5em}

    \begin{flushleft}
    Автор\hspace{.7cm}\underline{Байдаченко~В.~А.}\hfill\underline{\hspace{5cm}}\\
    \hspace{2cm}\footnotesize(Фамилия, имя отчество)\hspace{8cm}(Подпись, дата)\normalsize\\
    \vspace{.3cm}
    Группа Ф-469\\
    \vspace{.3cm}
    Направление 010700.62\\
    \vspace{.3cm}
    Руководитель работы (проекта)\\
    \underline{д. ф.-м. н., профессор Смоляр~В.~А.}\hfill\underline{\hspace{5cm}}\\
    \hspace{.4cm}\footnotesize(Уч. звание, уч. степень, фамилия, инициалы)\hspace{6.5cm}(Подпись, дата)\normalsize\\
    \vspace{.3cm}
    Консультанты\\
    по научной части \underline{\hspace{7cm}}\hfill\underline{\hspace{5cm}}\\
    \hspace{4cm}\footnotesize(Уч. звание, уч. степень, фамилия, инициалы)\hspace{3cm}(Подпись, дата)\normalsize\\
    \vspace{.3cm}
    Нормоконтроллер \underline{\ \ доц., к. ф.-м. н., Жуков~С.~С.\ \ }\hfill\underline{\hspace{5cm}}\\
    \hspace{4cm}\footnotesize(Уч. звание, уч. степень, фамилия, инициалы)\hspace{3cm}(Подпись, дата)\normalsize\\
    \end{flushleft}

    \vspace{\fill}

    \begin{center}
    Волгоград 2013
    \end{center}

    \end{titlepage}
\begin{titlepage}
\begin{center}
Министерство образования и науки РФ \\
\vspace{.5cm}
ГОСУДАРСТВЕННОЕ ОБРАЗОВАТЕЛЬНОЕ УЧРЕЖДЖЕНИЕ\\*
ВЫСШЕГО ПРОФЕССИОНАЛЬНОГО ОБРАЗОВАНИЯ\\*
<<ВОЛГОГРАДСКИЙ ГОСУДАРСТВЕННЫЙ ТЕХНИЧЕСКИЙ УНИВЕРСИТЕТ>>\\
\vspace{.5cm}
Кафедра <<ФИЗИКА>>
\vspace{.5cm}
\end{center}
\begin{flushright}
УТВЕРЖДАЮ\\
Заведующий кафедрой <<Физика>>\\
проф., д-р физ.-мат. наук А.~Г.~Шеин\\
\vspace{1cm}
<<\underline{\ \ \ \ }>>\underline{\ \ \ \ \ \ \ \ \ \ \ \ \ \ } 2013г.
\end{flushright}
\begin{center}
\large Задание \\
\normalsize на дипломную работу (проект)
\end{center}
\begin{flushleft}
Студент \hrulefill\\
Код кафедры \underline{\hspace{3cm}}\hspace{6cm}Группа Ф-469\\
Тема \hrulefill\\
\hrulefill\\
Утверждена приказом по ВолгГТУ\\
от ``\underline{\hspace{2cm}}''\underline{\hspace{5cm}} 2012г. \No \underline{\hspace{3cm}}\\
Срок предъявления готовой работы ``\underline{\hspace{2cm}}''\underline{\hspace{5cm}} 2013г.\\
Исходные данные для выполнения работы\\
\hrulefill\\
\hrulefill\\
Содержание основной части пояснительной записки

Перечень графического материала

Консультанты по разделам:
\begin{tabular}{|*{4}{C{.223}|}}\hline
\multicolumn{1}{|c|}{\multirow{2}{*}{Раздел}} &
\multicolumn{1}{c|}{\multirow{2}{*}{Консультант}} & 
\multicolumn{2}{c|}{Подпись, дата} \\ \cline{3-4}
&&Задание выдал&Задание принял\\ \hline
&&&\\ \hline &&& \\ \hline 
\end{tabular}
Задание выдал\\
Руководитель работы (проекта) \underline{\hspace{5cm}}\hfill\underline{\hspace{5cm}}\\
\hspace{14cm}\footnotesize(Подпись, дата)\normalsize\\
Задание получил \underline{\hspace{8.15cm}}\hfill\underline{\hspace{5cm}}\\
\hspace{14cm}\footnotesize(Подпись, дата)\normalsize\\
\end{flushleft}

\end{titlepage}% это титульный лист
\begin{center}
РЕФЕРАТ
\end{center}

Целью данной работы является изучение явления эффекта близости в электронной литографии низких энергий.
Его проявление в случае облучения электронным потоком слоя резиста ПММА на кремниевой подложке. В процессе работы были проведены расчеты по определение параметров поглощения энергии, обратного рассеяния, и диффузии на различных глубинах и при различной плотности энергии пучка.
\vspace*{1cm}

\hspace{-1.25cm}электронная литография, ПММА, метод Монте-Карло, эффект близости, диффузионные электроны, резист, обратно рассеянные электроны.
 \vspace*{1cm}

The purpose of this operation is the study of the phenomenon of effect of closeness in electronic lithograph of low energies.
Егоя manifestation in case of radiation by an electronic flow of a layer of PMMA resist on the Silicon substrate. In the course of operation calculations on determination of parameters of a poglavshcheniye of energy, back scattering, and diffusion at rvzlichny depths were carried out and in case of different density of energy of a bundle.
\vspace*{1cm}

\hspace{-1.25cm}electronic lithograph, PMMA, Monte-Carlo method, effect of closeness, diffusion electrons, resist, back scattered electrons.

\pagestyle{empty}
\renewcommand{\contentsname}{\begin{center}Содержание\end{center}}
\tableofcontents % это оглавление, которое генерируется автоматически
% \begin{center}
ОПРЕДЕЛЕНИЯ, ОБОЗНАЧЕНИЯ И СОКРАЩЕНИЯ
\end{center}

В настоящей работе применяют следующие термины с соответствующими определениями:
СЭМ -- сканирующий электронный микроскоп\\
МНП -- метод непрерывных потерь\\
ПММА -- полиметилметакрилат\\
$e$ -- заряд электрона\\
$k$ -- постоянная Больцмана\\
$d_c$ -- диаметр кроссовера\\
$d_p$ -- диаметр конечного пятна\\
$D^0$ -- чувствительность поглощения\\
$T^0$ -- чувствительность экспонирования\\
$\eta$ -- вязкость проявителя\\
$Q$ -- коэффициент диффузии проявителя\\
$\Sigma$ -- толщина диффузионного слоя\\
$\gamma$ -- контрастность\\
$v$ -- скорость проявления\\
$N_f$~-- плотность потока электронов, сохраняющих направление\\
$N_d$~-- плотность потока диффундирующих электронов\\
$W_{tr}$~-- транспортная частота столкновений\\
$\bar{\varepsilon}$~-- потеря энергии на единицу пробега\\
$W_{sa,tr}$~-- малоугловой компонент частоты столкновений\\
$W_d$ -- плотность поглощенной энергии диффузного компонента

\pagestyle{plain}
\chapter*{Введение}
\addcontentsline{toc}{chapter}{Введение}
\vspace*{1cm}

Миниатюризация элементов интегральных схем есть способ увеличения их
производительности и эффективности. Поэтому существует потребность как
в развитии наноструктурирования, так и создания специализированных
структур на чипах с элементами, имеющими нанометровые размеры в таких
областях как: оптоэлектроника, рентгеновская оптика, исследования в
области физики низких температур и квантово-размерных эффектов, новых
материалов, таких как двумерный графен и т.д.
Структурирование с помощью электронной литографии является самым
удобным методом создания объектов ввиду своей гибкости и оперативности.

\newpage
\chapter{Условия, от которых зависит разрешение литографии}
\section{Формирование пучка в электронном микроскопе}

Электронно-оптическая система электронного микроскопа конструируется таким образом, чтобы получать минимально возможный размер пучка электронов, при этом учитывая максимально возможной ток. На рис.~\ref{fig:1}. представлена общая схема электронного микроскопа

\begin{figure}[H]
\center
\includegraphics[scale=0.6]{1_1.pdf}
\caption{Общая схема сканирующего электронного микроскопа: 1 - электронная пушка; 2 - камера электронной пушки; 3 - анод; 4 - вакуумная магистраль; 5 - котировочные катушки; 6 - конденсорные линзы; 7 - объектная линза; 8 - детектор упруго отражённых электронов; 9 - система позиционирования образцов; 10 - держатель образцов; 11 - основная вакуумная камера; 12 - сканирующие катушки; 13 - рентгеновский спектрометр.}
\label{fig:1}
\end{figure}

Электронная пушка является стабильным источником электронов и используется для формирования электронного пучка. На рис.~\ref{fig:2} представлена общая схема электронной пушки на вольфрамовом катоде.

\begin{figure}[H]
\center
%\includegraphics[scale=0.6]{}
\caption{ Схематическое изображение электронной пушки и траектории электронов в ней: 1 -катод* 2-модулятор (цилиндр Венельта), 3- эквипотенциали, 4-анод, 5-сопротивление смещения, 6-источник высокого напряжения. $\alpha_0, d_0$ - угол расхождения электронов и диаметр}
\label{fig:2}
\end{figure}

Катод, цилиндр Венельта и анод образуют трёхэлектродную пушку. В пространстве между цилиндром Венельта и анодом пучок сужается, образуя кроссовер.
Диаметр кроссовера определяется тепловым разбросом скоростей и фокусным расстоянием пушки.

\begin{equation}
d_c \approx 2f \sqrt{\frac{kT_k}{eV_a}}
\label{eq:A1}
\end{equation}

Важной характеристикой электронной пушки является яркость. Она определяется как плотность тока в единице телесного угла.

\begin{equation}
\beta = \frac{\Delta I}{\Delta S\Delta\Sigma}= \frac{i}{\pi \alpha^2}= const
\label{eq:A2}
\end{equation}

Яркость остаётся постоянной для всех точек на оси электронно-оптической системы от катода до конечного пятна пучка независимо от диафрагм и электронных линз в электронной оптике. Теоретически достижимое значение яркости:

\begin{equation}
\beta_{max} \simeq \frac{i_c E}{\pi kT_c}
\label{eq:A3}
\end{equation}

На рис.~\ref{fig:3} схематически представлена электронно-оптическая система, которая создает уменьшенное изображение кроссовера.

\begin{figure}[H]
\center
%\includegraphics[scale=0.6]{}
\caption{Схематический ход лучей в электронно-оптической колонне типичного электронного микроскопа.}
\label{fig:3}
\end{figure}

В случае идеальной оптической системы получение пучка минимального размера ограничивалось бы получением уменьшенного изображения кроссовера. Но, в реальных системах имеют место факторы, приводящие к уширениям минимального размера, такие как аберрации, астигматизм и дифракция.

Сферические аберрации дают уширение

\begin{figure}
    \includegraphics[width=.47\textwidth]{}\hfill
    \includegraphics[width=.47\textwidth]{}
    \parbox[t]{.47\textwidth}{\caption{раз}}\hfill
    \parbox[t]{.47\textwidth}{\caption{два}}

    \includegraphics[width=.47\textwidth]{}\hfill
    \includegraphics[width=.47\textwidth]{}
    \parbox[t]{.47\textwidth}{\caption{три}}\hfill
    \parbox[t]{.47\textwidth}{\caption{четыре}}

\end{figure}

Для устранения этих эффектов используются электронные линзы специальной конфигурации.
Формирование конечного пятна происходит последовательным уменьшением кроссовера в 2-3 электронных линзах.

\begin{equation}
d_0= \frac{f_1 f_2 f_3}{L_1 L_2 L_3} d_c = md_c
\label{eq:A4}
\end{equation}

Эффективный диаметр конечного электронного пятна задаётся

\begin{equation}
d_p^2=d_0^2+d_d^2+d_s^2+d_c^2
\label{eq:A5}
\end{equation}

Слагаемыми ответственными за дифракцию и хроматическую аберрацию можно пренебречь ввиду их малости. Таким образом, решая задачу по определению наилучшего соотношения размера конечного пучка электронов к току, получается

\begin{equation}
\alpha_{opt}\sim{(}C_0/C_s{)}^{1/4},\>\>d_{p,min}\sim{(}C_0^3 C_s{)}^{1/4},\>\> I_{p,max}\sim \beta C_s^{-2/3} d_{p,min}^{8/3}\label{eq:A6}
\end{equation}
Здесь
\[
    d_0= \left(\frac{4I_p}{\pi^2 \beta}\right)^{1/2} \alpha_p^{-1}=C_0\alpha_p^{-1},
    \quad j_p=\pi \beta \alpha_p^2 , I_p= \frac{\pi d_0^2 j_p}{4}.
\]

Таким образом, при неизменном токе
\begin{equation}
d_{p,min}\sim (eU)^{-3/8},d_{p,min}\sim(eU)^{-1/2}
\label{eq:A7}
\end{equation}
в идеальной оптической системе
\begin{equation}
d_{p,min}\sim (eU)^{-3/8},d_{p,min}\sim(eU)^{-1/2}
\end{equation}

\section{Рассеяние электронов и эффекты близости}

Основной проблемой электронной литографии является эффект близости, который заключается в рассеянии электронов в резисте и подложке и экспонировании ими резиста. На рис.~\ref{fig:4} представлены траектории рассеяния электронов в резиста на кремниевой подложке для энергий 10 кВ и 20 кВ, полученные результат моделирования методом Монте-Карло в работе [4]

\begin{figure}[H]
\center
%\includegraphics[scale=0.6]{}
\caption{ Расчет траекторий рассеяния электронов в пленке резиста на кремниевой подложке, выполненный методом Монте-Карло, для энергии электронов 10кВ (слева) и 20кВ (справа)}
\label{fig:4}
\end{figure}

Как видно из рис.~\ref{fig:4}, область экспонирования обратно рассеянными электронами гораздо больше диаметра электронного луча и составляет микроны, зависит от ускоряющего напряжения и от типа подложки. Длина пробега при рассеянии пропорциональна $E^{1.7}$, где Е- энергия электронов в падающем луче
В зависимости от отсутствия или наличия ближайших соседей наблюдается соответственно внутренний или взаимный эффект близости . Внутренний эффект близости, обусловленный обратным рассеянием электронов за

пределы непосредственно экспонированной области, приводит к тому, что уединенные мелкие элементы топологии проявляются гораздо медленнее, чем большие фигуры. Взаимный эффект близости заключается в экспонировании ближайшими соседями друг друга и пространства между ними. Неэкспонированные области между линиями засвечиваются обратно рассеянными электронами.
Эффект близости в электронной литографии количественно описывается функцией близости
Функция близости - это распределение энергии (дозы) экспонирования, поглощенной в резисте. Она является суммой двух гауссовых функций, описывающих вклады в поглощенную дозу первичных электронов
Для решения проблемы коррекции эффекта близости используются несколько способов. Наиболее распространенным способом является коррекция дозы экспонирования. Доза экспонирования каждого элемента структуры устанавливается с учетом влияния эффекта близости так, чтобы в результате проявления все элементы структуры проявились одновременно и с точными размерами.

Эффект близости в электронной литографии количественно описывается функцией близости

\begin{equation}
I(\rho)=\frac{I_1 (\rho)+ \eta I_2(\rho)}{1+\eta}=\frac{1}{\pi(1+\eta)}\left[\frac{1}{\alpha^2}\exp\left(-\frac{\rho^2}{\alpha^2}\right)+\frac{\eta}{\rho^2}\exp\left(-\frac{\rho^2}{\beta^2}\right)\right],\quad \rho^2 = x^2 + y^2,
\label{eq:A8}
\end{equation}

Функция близости - это распределение энергии (дозы) экспонирования, поглощенной в резисте. Она является суммой двух гауссовых функций, описывающих вклады в поглощенную дозу первичных электронов
$\frac{1}{\pi(1+\eta)} \frac{1}{\alpha^2} \exp{(}-\frac{\rho^2}{\alpha^2}{)} $ и обратно отраженных $\frac{1}{\pi(1+\eta)} \frac{1}{\rho^2} \exp {(}-\frac{\rho^2}{\beta^2}{)}$. 
Физический смысл параметров $\alpha,\beta,\eta$ следующий: $\alpha$ - размер первичного пучка с учетом рассеяния его в резисте при прямом прохождении, причем считается, что плотность электронов в пучке имеет гауссово распределение, и диаметр пучка ограничен областью, где плотность электронов в $e$ раз меньше чем в центре пучка; $\beta$ - характерный размер области рассеяния электронов в материале подложки; $\eta$ - интегральная характеристика вклада, который вносят обратно рассеянные электроны в суммарную дозу.

Энергия (доза) поглощенная резистом определяется как свертка дозы экспонирования и функции близости

\begin{equation}
\frac{D(x,y)}{D^0}= \int \frac{I(x-x',y-y')T(x',y')}{T^0} dx' dy'
\label{eq:A9}
\end{equation}

$D^0,T^0$ - чувствительности поглощения и экспонирования соответственно.
Метод простых компенсаций реализуется следующим образом. Происходит разбиение структуры на области с постоянной дозой экспонирования. Затем методом итераций, используя вышеупомянутые формулы, рассчитывается средняя доза для каждой области с учётом дозы соседних областей. В электронной литографии экспонирование областей и линий производится по точкам. “Попав’ в некоторую точку, электронный луч задерживается в ней на рассчитанный интервал времени для сообщения резисту необходимой дозы. Потом он перемещается в следующую точку, и т.д. Временем перехода между точками можно пренебречь, так как оно много меньше времени, которое пучок затрачивает на точку. Интервал рассчитывается программой исходя из заданной дозы и шага между точками. Следовательно, фиксированная доза при экспонировании области означает, что в пределах этой области электронный зонд будет задерживаться в каждой точке на одинаковое время.

Чтобы коррекция была успешной важно знать параметры $\alpha,\beta,\eta$. Существующая серия тестов позволяет достаточно точно определить эти параметры Из того, что было выше сказано об этих параметрах, становится очевидно, какие именно условия проведения экспонирования оказывают на них влияние: $\alpha$ - не зависит от материала подложки и определяется ускоряющим напряжением, толщиной резиста и качеством настройки фокуса микроскопа (опытом оператора); $\eta$ - параметр, определяющийся материалом подложки, $\beta$ - материалом подложки и ускоряющим напряжением.

Метод простых компенсаций позволяет точно скорректировать вклад обратно рассеянных электронов, однако, эффект первичного пучка не может быть устранен полностью, а это означает, например, что невозможно создать два элемента, расстояние между которыми было бы меньше размера первичного пучка а. Таким образом, влияние первичного пучка может быть скорректировано только приблизительно.

\section{Взаимодействие электронов с резистом}

Поглощение излучения высоких энергий происходит в результате взаимодействия падающих лучей с электронами в атомах. Электроны теряют энергию под действием фотоэффекта путем возбуждения атомных электронов.
Электронный пучок, при плотностях тока используемых в современных установках (до 10 А/см в электронной литографии, в СЭМ плотности тока существенно меньше), можно считать разреженным, т.е. электроны пучка не взаимодействуют между собой в процессе рассеяния в образце, поэтому рассеяние каждого электрона можно рассматривать по отдельности.
При моделировании прохождения электрона через вещество также можно считать, что большую часть времени электрон свободно летит, скачкообразно изменяя направление полёта и/или энергию в процессе упругого или неупругого взаимодействия с атомами вещества. Вероятность рассеяния электрона в веществе описывается сечением рассеяния, которое вычисляется отдельно для каждого типа взаимодействия.
Таким образом, материал образца, в котором рассматривается поведение электрона, представляется смесью составляющих его атомов. Такое приближение верно лишь для аморфных и (в некотором смысле) поликристаллических веществ, не позволяя рассматривать эффекты специфичные для рассеяния в кристаллических телах, например каналирование (изменение тормозной способности в зависимости от направления движения электрона).
Существует несколько подходов к моделированию потерь энергии в процессе прохождения электронов сквозь вещество. Наиболее строгий подход к проблеме взаимодействия электронов с мишенью заключается в использовании модели однократного рассеяния	в сочетании с
одной или несколькими из моделей торможения, учитывающей потери энергии первичных и вторичных частиц за счёт наиболее высокоэнергетических механизмов.
В этой модели, в частности, вычисляют средний свободный пробег частицы между двумя столкновениями и используют далее выражения сечения рассеяния для розыгрыша угла отклонения в процессе взаимодействия. Число соударений, испытываемых электроном вдоль траектории в среднем более 104, поэтому модельные расчёты методом Монте-Карло, основанные на учёте однократных процессов рассеяния требуют огромных затрат машинного времени.
Метод непрерывных потерь (МНП) - одна из разновидностей метода однократного рассеяния наиболее часто применяемая и простая для реализации. В качестве шага алгоритма берётся расстояние между упругими соударениями. Считается, что между ними электрон равномерно теряет энергию, количество которой определяется тормозной способностью вещества.
Первое универсальное выражение для дифференциальных потерь энергии электронов было получено Бёте

\begin{equation}
-\frac{dE}{dx}=\frac{2\Pi e_e^4 NZ}{E} \ln {(}\frac{E}{2CZ}b{)}
\label{eq:A10}
\end{equation}

Здесь $Е$ - энергия налетающего электрона, $e_e$ - заряд электрона, $Z$ - атомный номер, $С$ - эффективная энергия ионизации. Для константы b предсказываются значения $b = 1$ в классической теории. Неупругие каналы рассеяния учитываются при помощи введения эффективной энергии ионизации С. Этот параметр был измерен экспериментально методами атомной физики. В качестве аналитической аппроксимации была предложена формула:

\begin{equation}
C(Эв)=Z{(}9.76+58.8Z^{-1.19}{)}
\label{eq:A11}
\end{equation}

Следовательно, область применимости формулы Бете лежит в области энергий частиц $\gg С$  (-400-800 эВ, в зависимости от материала образца). Закон Бете даёт хорошо согласующееся с экспериментом значение тормозной способности. Вычисляемые с его помощью потери энергии включают в себя все потери, в том числе и связанные с генерацией вторичных электронов.
Рассмотрим формулу Бёте (10) пренебрегая изменениями логарифма в толще резиста ввиду малости изменения энергии пучка при прохождении пленки резиста мало, получим:

\begin{equation}
\Delta E/h_0 \infty 1/E= 1/(eV)
\end{equation}

где $h_0$ - толщина резиста. Таким образом, энергия, воспринятая резистом, обратно пропорциональна ускоряющему напряжению. Соответственно чувствительность резиста $S$ будет пропорциональна $V$ (см. Гл. 1, п. 5)

\begin{equation}
S \sim V
\end{equation}

\section{Проявление полимерных резистов. Основные законы проявления.}

В основном, как позитивные, так и негативные электронные резисты являются полимерами. При взаимодействии полимерных резистов с электронным пучком у них может происходить структурирование или деструкция полимерной цепи, либо, как это происходит у большинства полимеров, оба процесса идут одновременно. В случае преобладания структурирования резист является негативным, а деструкции позитивным. Скорость проявления резиста зависит многих факторов, в том числе от условий сушки, концентрации, типа проявителя и его температуры [14]. Рассмотрим подробнее процесс проявления полимерного резиста [15]. В самом общем случае проявление описывается законом Фика для диффузии или законом Грюнера для растворения твердого вещества. На рисунке 1.5 дана картина распределения молекул полимера, растворяющегося без набухания (таким полимером, например, является ПММА). Растворение идет до тех пор, пока не насыщается слой жидкости, непосредственно примыкающий к полимеру. Растворенные молекулы полимера диффундируют из насыщенного слоя в соседние слои с меньшей концентрацией. Таким образом, скорость растворения есть по сути дела скорость самой диффузии.
Коэффициент диффузии проявителя $Q$ (около $10^{-6}-10^{-8} см^2/c$) связан с вязкостью проявителя $\eta$ соотношением Стокса - Эйнштейна.

\begin{equation}
Q=RT/\eta=RT / KM^{\alpha}
\label{eq:A12}
\end{equation}

где $KM^a$ - зависимость вязкости (Марка-Хаувинк) от молекулярной массы ($M$). При изотропном растворении за 1 с молекула переместиться в среднем на 

\begin{equation}
\sqrt{Qt}=10^{-3}~\text{см}.
\label{eq:A13}
\end{equation}

Энергия активации для диффузионно-контролируемого проявления имеет величину порядка 2-8 ккал/моль {[}16{]}
Если жидкий проявитель энергично перемешивается, то диффузия молекул от поверхности твердой фазы ускоряется, но тонкий градиентный слой толщиной 100-10 мкм при этом прочно держится у поверхности. Если £ - толщина статического слоя (рис.~\ref{fig:5}). $A_s$ - концентрация при насыщении и $С$ - концентрация растворенного полимера, однородная из-за перемешивания, то закон Фика для растворения массы полимера $dm$ за время $dt$ с поверхности $S$ имеет вид

\begin{equation}
dm=Q\frac{A_s -C}{\Sigma} SMdt
\label{eq:A14}
\end{equation}

Толщина слоя $\Sigma$ убывает с улучшением перемешивания, и общая скорость растворения от этого возрастает(рис.~\ref{fig:5})

\begin{figure}[H]
\center
%\includegraphics[scale=0.6]{}
\caption{Градиент концентрации вблизи поверхности проявляемого резиста.}
\label{fig:5}
\end{figure}

Прирост концентрации $dC$ зависит от объема жидкости $V$ и молекулярной массы полимера $М$

\begin{equation}
(dm/M)(1/V)-dC
\label{eq:A15}
\end{equation}

Интегрирование уравнения (14) дает

\begin{equation}
\ln((A_s-C)/A_s) -Qst/V\Sigma
\label{eq:A16}
\end{equation}

что эквивалентно реакции первого порядка. Подстановка плотности $р$ и толщины $dz$ в выражение для массы $dm$ или концентрации $dC$ дает экспоненциальную зависимость для скорости растворения

\begin{equation}
\frac{dz}{dt} = A_s MQ \rho Q \exp\left(\frac{-QSt}{MV}\right)
\label{eq:A17}
\end{equation}

Закон Фика, предполагающий, что коэффициент диффузии $Q$ проявителя не зависит от молекулярного веса полимера, не выполняется для многих стекловидных полимеров, в которых набухание препятствует растворению.
В общем случае из закона Фика видно, что скорость растворения полимера обратно пропорциональна его молекулярной массе В резистах наблюдаются два основных типа растворения:
\begin{enumerate}
    \item растворение экспонированного и неэкспонированного резиста практически линейно, если поглощение света или другого вида энергии в пленке однородно по толщине (рис.~\ref{fig:6}а). Это наблюдается в низко молекулярном ПММА ($М_n<10^4$). При ЭЛ-экспонировании поглощение энергии в тонких пленках (<300 нм) однородно, а в более толстых пленках возрастает с глубиной из-за большего вклада обратного рассеяния. Случай 1 относится к простому послойному растворению пленки.
    \item растворение неэкспонированных участков резиста замедляется в меру длительности индукционного периода (рис.~\ref{fig:6}6)  Случай 2 учитывает проникновение проявителя в пустоты, но быстрого растворения не происходит из-за образования набухающего гель-слоя или поверхностной адсорбции продуктов растворения. Образование нескольких физических
\end{enumerate}

\begin{figure}[H]
\center
%\includegraphics[scale=0.6]{}
\caption{А) линейно растворяющийся полимер, В) нелинейно растворяющийся резист (индукционный эффект): А - неэкспонированный резист, Б - экспонированный резист с малой дозой, В - экспонированный резист с большой дозой}
\label{fig:6}
\end{figure}

слоев при растворении поверхности полимера было промоделировано Юберейтером с сотрудниками Растворение может также сопровождаться шелушением или растрескиванием, ослабляющим внутренние напряжения при набухании Образование гель-слоя очень выгодно при наличии индукции, так как растворение неэкспонированного резиста задерживается на несколько минут, в течение которых изображение частично проявляется.

\section{Характеристики резиста}

Для успешного применения резистов на практике, необходимо знание нескольких параметров, называемых еще дозовыми характеристиками. Дозовыми характеристиками электронных резистов являются чувствительность и контрастность. Чувствительность позитивного резиста — это минимальная необходимая для проявления резиста до дна доза экспонирования Чувствительность позитивного резиста характеризует дозу, которую необходимо передать участку этого резиста для его полной проявки за приемлемое время (обычно 1-2 минуты). Чувствительность, как и дозу экспонирования электронного резиста обычно измеряют в $\text{Кл}/\text{см}^2$ . Для
определения чувствительности обычно используют тестовую структуру, состоящую из массива одинаковых элементов (например, квадратов) с
различной дозой экспонирования. Задаётся определённый шаг дозы от одного элемента к следующему. После экспонирования и проявления тестовой структуры, для позитивного резиста ищут среди проявившихся до дна элементов, тот который был проэкспонирован с минимальной дозой. Эту дозу и принимают равной чувствительности позитивного резиста. В настоящей работе в качестве такой структуры выступает дозовый клин, доза экспонирования которого меняется слева на право в 10 раз.
Другой важной характеристикой электронного резиста является контрастность. Она является характеристикой крутизны рельефа резиста после проявления. В литературе контрастность для позитивного резиста определяют несколькими способами	В качестве контрастности для
позитивного резиста, можно предложить использовать параметр растворимости. Действительно, обратная молекулярная масса полимерного резиста зависит от дозы экспонирования. 
Таким образом, можно записать, что

\begin{equation}
D\approx \frac{1}{M}
\label{eq:A18}
\end{equation}

Скорость проявления линейного растворяющегося резиста может быть описана степенной зависимостью [16,22,23]

\begin{equation}
\nu=aM^{-\gamma}
\label{eq:A19}
\end{equation}

Тогда используя формулы (18) и (19) можно получить зависимость скорости проявления резиста от дозы экспонирования 

\begin{equation}
\nu=\nu_0\left(\frac{D}{D_0}\right)^\gamma,
\label{eq:A20}
\end{equation}

где $v$ это скорость проявления области проэкспонированной с дозой $D$, а $v_0$ -- скорость проявления области проэкспонированной с дозой $D_0$. параметр $\gamma$ принимается за контрастность. 


\newpage
\chapter{Математическая модель}
Задачи о транспорте быстрых электронов в многослойных мишенях в настоящее
время решаются почти исключительно методом Монте-Карло, феноменологические
модели здесь практически непригодны, а аналитические методы решения кинетического
уравнения недостаточно разработаны даже для однородных слоев и ограничиваются
частными задачами — это либо задача об обратном рассеянии, либо, что значительно
реже, задача об энерговыделении для широкого пучка, падающего на однородную
мишень. Известным нам исключением является работа \cite{zheng-ming}, в которой двухгрупповая
модель расщепления кинетического уравнения Спенсера была применена для расчета
энерговыделения широкого пучка в двухслойной мишени. Однако в этой работе
уравнение Спенсера записано для плотности потока электронов в фазовом пространстве,
включающем остаточный пробег электронов, и поэтому оно неприменимо для
неоднородных сред, так как в таких средах нет однозначного соответствия между
энергией электронов и их остаточным пробегом и пробег не имеет физического смысла.
Для многослойных мишеней следует применять кинетическое уравнение для электронов в
фазовом пространстве. Включающем энергию, как это сделано в работе \cite{smolar}.
В наших предыдущих работах \cite{mikheev,smolar1,smolar2} было описано расщепление транспортного
уравнения переноса электронов на основе соединения в единой математической модели
двух предельных приближений — малоуглового и транспортного. В этой модели частота
упругих столкновений аппроксимируется суммой изотропного и малоуглового
компонентов, а плотность потока электронов представляется в виде суммы двух
компонентов — сохраняющего первоначальное направление и диффундирующего.
В случае нормального падения на образец, состоящий из n слоев, точечного пучка
электронов с начальной энергией $E_0$ не происходит слишком большого усложнения
задачи, и она остается в пределах возможностей применяемых аналитических и
численных методов. Плотность потока электронов в среде также представляется в виде
суммы
(1)
\begin{equation}
N(z,\vec{r},\vec{\Omega},E)=N_f (z,\vec{r},\vec{\Omega},E)+N_d (z,\vec{r},\vec{\Omega},E),
\label{eq:1}
\end{equation}
где $N_f$ и $N_d$ – плотности потока электронов, сохраняющих направление и
диффундирующих, ось $z$ с началом в точке падения пучка направлена в глубину мишени,
$\vec{r}$ – вектор, ортогональный оси $z$, $\vec{\Omega}$ – направление скорости и $E$ – энергия электронов.
Для функции $N_f$ было получено аналитическое решение в однородном слое \cite{smolar2} и в
многослойной структуре \cite{smolar}. Здесь мы запишем аналитическое решение для $N_f$ в
многослойной мишени в несколько иной форме. Транспортная частота и средние потери
энергии в такой мишени являются ступенчатыми функциями относительно переменной $z$
и могут быть представлены с помощью характеристических функций $\chi_i(E)$ i-того слоя –
произведения единичных ступенчатых функций Хевисайда $U(x)$


\begin{equation}
\bar{\varepsilon}(E,z)=\sum_{i=1}^n \chi_i (E)\bar{\varepsilon}^{(i)} (E),
\label{eq:2}
\end{equation}

\begin{equation}
W_tr(E,z)=\sum_{i=1}^n \chi_i (E)W_{tr}^{(i)} (E),
\label{eq:3}
\end{equation}

\begin{equation}
W_{sa,tr}(E,z)=\sum_{i=1}^n \chi_i (E)W_{sa,tr}^{(i)} (E),
\label{eq:4}
\end{equation}
где
$ \chi_i(E)=U(E_{i-1}-E)U(E-E_i) $,
$Ei$ – энергия прямоидущих электронов на правой границе i-того слоя, $W_tr(E,z)$ –
транспортная частота столкновений, $\bar{\varepsilon}(E,z)$ – потеря энергии на единицу пробега, $W_{sa,tr}(E,z)$ – малоугловой компонент частоты столкновений.
Плотность малоуглового компонента потока имеет вид \cite{smolar}

%эту формулу в 3 этажа можно ибо не влезает в лист
\begin{equation}
N_f(z,\vec{r},\vec{\Theta},E)=\frac{\delta[z-s(E,z)]}{[\pi \bar{\varepsilon}(E,z)\sigma^2(z)]}\times \exp \frac{[\frac{(-\vec{\Theta}^2 C_2(z)-\vec{r}^2C_0 (z)+2(\vec{r}\vec{\Theta})C_1(z))}{\sigma^2(z)}]}{[\pi\sigma^2(z)]} \times \exp\left(-\int_{E}^{E_0} \frac{w_{tr}(E',z)}{\bar{\varepsilon}(E',z)}dE'\right),
\label{eq:6}
\end{equation}

где

\begin{equation}
C_k(z)=2 \int_{E}{E_0} \frac{w_{sa,tr}(E',z)[z-s(E',z)]^k}{\bar{\varepsilon}(E',z)} dE',\>\> к=0,1,2.,
\label{eq:7}
\end{equation}

\begin{equation}
\sigma^2(z)=C_0(z)C_2(z)-C_1^2(z),
\label{eq:8}
\end{equation}

\begin{equation}
s(E,z)=\sum_{i=1}^{n} \int_{E}^{E_0} \frac{\chi_i}{\bar{\varepsilon}(E',z)} dE',
\label{eq:9}
\end{equation}

Функция $s(E, z)$ имеет смысл среднего пути электрона, сохраняющего первоначальное
направление вдоль оси $z$, когда его энергия уменьшается от значения $E_0$ в точке $z=0$ до $E$
на глубине $z$.
Плотность потока диффузного компонента $N_d(z,\vec{r},\vec{\Omega},E)$ является решением
уравнения \cite{smolar}

\begin{gather}
-\frac{\partial}{\partial E}[\bar{\varepsilon}(E,z)N_d(z,\vec{r},\vec{\Omega},E)]+\vec{\Omega} \bigtriangledown N_d (z,\vec{r},\vec{\Omega},E)= \nonumber \\
\frac{1}{4\pi} \int d\vec{\Omega}'W_{tr}(E,z)[N_d(z,\vec{\Omega}',E)-N_d(z,\vec{r},\vec{\omega},E)]+\frac{1}{4\pi}w_{tr}(E,z) \int d\vec{\Omega}'N_f(z,\vec{r}',\vec{\Omega}',E),
\label{eq:10}
\end{gather}


Используя аналитическое решение (6) и численное решение уравнения (10), мы
можем определить любые дифференциальные и интегральные характеристики переноса
электронов в мишени. Расчеты распределений обратно рассеянных и прошедших
электронов обсуждались ранее \cite{smolar}. В настоящей работе мы рассмотрим распределение
плотности поглощенной энергии электронов.
Учитывая аксиальную симметрию задачи, представим плотность поглощенной
энергии в виде

\begin{equation}
W(z,r)=\int_{0}^{E_0} dE\bar{\varepsilon}(E,z) \int d\vec{\Omega}E(z,\vec{r},\vec{\Omega},E)=W_f(z,r)+W_d(z,r),
\label{eq:11}
\end{equation}
где

\begin{equation}
W(z,r)=\int_{0}^{E_0} dE\bar{\varepsilon}(E,z)2\pi \int_{-1}{1} d(\cos \Theta)N_f(z,r,\Theta ,E),
\label{eq:12}
\end{equation}

\begin{equation}
W_{d}(z,r)=\int_{0}^{E_0} dE\bar{\epsilon}(E,z)2\pi \int_{-1}^{1} d(\cos \Theta)N_d(z,r,\Theta ,E)=\int_{0}^{E_0}dE\vec{\varepsilon}(E,z)N_{d0}(z,r,E),
\label{eq:13}
\end{equation}

\begin{equation}
N_{do}(z,r,E)=2\pi \int_{-1}{1} d(\cos \Theta)N_d(z,r,\Theta,E),
\label{eq:14}
\end{equation}

Здесь использована цилиндрическая система координат,$r=|\vec{r}|$ -- радиальное
расстояние, $\Theta$ — угол между осью $z$ и вектором $\vec{\Omega}$ .
Используя (6) и выполняя интегрирование в (12), получаем гауссиан
\begin{equation}
W_f(z,r)=\frac{A_f(z)}{\pi \sigma_f^2(z)} \exp \left(\frac{r^2}{\sigma_f^2(z)}\right),
\label{eq:15}
\end{equation}

с полушириной $\sigma_f(z)=\sqrt{C_2 (z)}$ и амплитудой
\begin{equation}
A_f(z) = \int\limits_0^{E_0}\delta( z − s(E, z))\exp\left(\int\limits_0^{E_0}\frac{w_{tr} ( E′, z)}{\bar{\varepsilon}(E′, z)}dE′\right) dE
\end{equation}
Используя представление двумерной \(\delta\)-функции с помощью последовательности непрерывных функций \cite{korn}
\begin{equation}
\delta(\vec{r}) = \lim_{\sigma\rightarrow0} \frac{\exp\left(−\frac{r^2}{\sigma^2}\right)}{\pi\sigma^2}
\end{equation}
из (16) получаем
\begin{equation}
\lim_{z\rightarrow0} W_f(z, r) = A_f(0)\delta(\vec{r}),
\end{equation}
так что распределение \( W_f(z, r) \) является сингулярным вблизи точки падения
электронного пучка. Отметим, что
\begin{equation}
    A_f(0) = \bar{\varepsilon}(E=0, z=0).    
\end{equation}
Плотность поглощенной энергии диффузного компонента \(W_d(z, r)\) может быть
рассчитана с помощью функции \(N_{d0}(z, r, E),\) которая является решением краевой задачи.
Введя для удобства новую переменную \(\tau=E_0-E\) -- остаточную энергию электронов,
представим задачу для функции \(N_{d0}(z, r, \tau)\) в стандартном виде [12]
\begin{align}
& \frac{\partial N_{d0}}{\partial\tau} = \frac{\partial}{\partial z}\left[A (\tau , z )\frac{\partial N_{d0}}{\partial z}\right]
+ \frac{1}{r}\frac{\partial}{\partial r}\left[rA(\tau, z)
\frac{\partial N_{d0}}{\partial r}\right] + B (\tau , z )
\frac{\partial N_{d0}}{\partial z} − G (\tau , z )N_{d0}
+ \\
& \exp\left( −\frac{r^2}{C_2(z)}\right)\pi C_2(z)\frac{g(\tau, z)}{\bar{\varepsilon} (\tau , z ) }\delta ( z − s(\tau , z )).
\nonumber
\end{align}
где
\begin{align}
& A (\tau, z) = \frac{1}{\bar{\varepsilon}(\tau, z)w_{tr} (\tau, z)}, \\
& B (\tau, z) = \frac{1}{w_{tr}(\tau, z)}\frac{\partial\frac{1}{\bar{\varepsilon}(\tau, z)}}{\partial z}, \\
& G (\tau, z) = \frac{\partial\ln\bar{\varepsilon}(\tau, z )}{\partial\tau},\\
& g(\tau, z) = \frac{w_{tr}(\tau , z )}{\bar{\varepsilon} (\tau , z )}\exp\int( −\int_0^{E_0} \frac{w_{tr}(\tau′, z )}{\bar{\varepsilon} (\tau′, z)}d\tau′.    
\end{align}
Для функций \( \tau (\tau , z ), w_{tr}(\tau , z)\) и \(s(\tau,z)\) используем аппроксимации, описанные в \cite{smolar3}.
Пусть \(\xi_1,\ldots,\xi_n\) -– точки разрыва функции \(\bar{\varepsilon}(\tau , z)\) по отношению к переменной \( z \) (эти точки являются границами слоев). Тогда
\begin{equation}
B(\tau, z) = −\frac{1}{w_{tr}(\tau, z)}\sum_{i=1}^{n} \bar{\varepsilon} −1 (\tau ,\xi_i ) \delta ( z − \xi i ) ,    
\end{equation}
где \(\bar{\varepsilon}^{−1}\) обозначает скачек
\begin{equation}
    \bar{\varepsilon}^{−1} (\tau ,\xi_i ) = \bar{\varepsilon}^{−1} (\tau,\xi_i + 0 ) − \bar{\varepsilon}^{−1} (\tau ,\xi_i − 0 ).
\end{equation}

В однослойной мишени имеем \( B(\tau,z)=0 \).
Уравнение (20) должно быть решено в области
\begin{equation}
    (z, r,\tau) \in [0, z_\text{макс}]\times [0, r_\text{макс} ]\times[0, E_0],    
\end{equation}
где величина \( z_\text{макс} \) определяется структурой мишени, а значение \( r_\text{макс} \) должно быть
конечным из вычислительных соображений. Представляется естественным выбрать \( r_\text{макс} \)
много большим, чем максимальная оценка длины свободного пробега в слоях.
Граничные и условия Маршака в стандартной форме \cite{keiz} имеют вид
\begin{align}
& \left[0,5 \frac{N_{d0}}{\bar{\varepsilon}(\tau, z)} − A (\tau , z)\frac{\partial N_{d0}}{\partial z}\right]_{z=0} = 0 ,\\
& \left[0,5 \frac{N_{d0}}{\bar{\varepsilon}(\tau, z)} − A (\tau , z)\frac{\partial N_{d0}}{\partial z}\right]_{z=z_\text{макс}} = 0 ,\\
& \left[0,5 \frac{N_{d0}}{\bar{\varepsilon}(\tau, z)} − A (\tau , z)\frac{\partial N_{d0}}{\partial r}\right]_{r=r_\text{макс}} = 0,\\
& \left[\frac{\partial N_{d0}}{\partial r}\right]_{r=0} = 0\\
& N_{d0}(z, r, \tau=0 ) = 0.
\end{align}

\newpage
\chapter{Экспериментальная часть}
Используя расчетную программу построенную на базе приведенной выше математической модели, было проведено моделирование нормального падения моноэнергетического пучка электронов на подложку с нанесенным на нее слоем резиста. Энергия электронов задана на уровне 20 кэВ, подложка выполнена из кремния толщиной 5000 нм, а резист выполнен из ПММА и имел толщину 500 нм.

\begin{figure}[h]
    \center
    \includegraphics[width=.47\textwidth]{example}
    \caption{Схематическое изображение подложки с нанесенным на нее резистом}
    \label{fig:example}
\end{figure}
В налетающем пучке распределение плотности энергии имеет вид, представленный на рисунке~\ref{fig:beam}.
\begin{figure}[h]
    \center
    \includegraphics[width=.95\textwidth]{beam}
    \caption{радиальное распределение плотности энергии в налетающем пучке}
    \label{fig:beam}
\end{figure}

Как можно видеть из графика, представленного на рис.~\ref{fig:fluxes}, при данном значении энергии лишь малая часть электронов отражается обратно, в основном они диффундируют. Что говорит о малом количестве электронов участвующих в эффекте близости.
% здесь можно приплести про эффект близости
\begin{figure}[h]
    \center
    \includegraphics[width=.95\textwidth]{fluxes}
    \caption{Доля потоков электронов:
    1~--~электроны, перешедшие из нерассеянных в диффузионные;
2~--~диффундировавшие электроны с энергией $E$;
3~--~отраженные электроны с энергией от $E_0$ до $E$;
4~--~прошедшие насквозь через подложку электроны с энергией от $E_0$ до $E$.
}
    \label{fig:fluxes}
\end{figure}

По мере прохождения доля диффузионных электронов растет.
\begin{figure}[h]
    \center
    \includegraphics[width=.95\textwidth]{rates}
    \caption{Соотношение диффузионных и нерассеянных электронов: 1~--~отношение потока нерассеянных электронов к потоку диффузионных;
2~--~доля потока нерассеянных электронов;
3~--~доля потока диффузионных электронов.
}
    \label{fig:rates}
\end{figure}
При прохождении в глубь резиста, а впоследствии и подложки, пучок уширяется
\begin{figure}[h]
    \center
    \includegraphics[width=.95\textwidth]{sequence}
    \caption{Поперечные размеры пучка:
1~--~полуширина пучка нерассеянных электронов;
2~--~полуширина пучка диффузионных электронов по 1-му гауссиану;
3~--~полуширина пучка диффузионных электронов по 2-му гауссиану;
4~--~полуширина пучка диффузионных электронов по 3-му гауссиану;
5~--~полуширина пучка диффузионных электронов по 4-му гауссиану.}
    \label{fig:sequence}
\end{figure}

Из зависимости показанной на рис.~\ref{fig:sequence} видно что пр прохождении пучка через резист на глубине примерно 230 нм находится максимум поглощения энергии пучка резистом и мишенью в целом. Следующий максимум поглощения находится в приграничной зоне подложки и резиста, далее интенсивность спадает. Данные результаты можно интерпретировать следующим образом. Пи использовании относительно низкой энергии электронного пучка при прохождении его через резист он начинает диффундировать, и в последствии поглощаться резистом, так как энергии электронов относительно не велики. При переходе в подложку, те электроны которые прошли через резист 2 на рис.~\ref{fig:fluxes} и не были поглощены но потеряли энергию в результате диффузии, при переходе в более плотную среду, поглощаются.
\begin{figure}[h]
    \center
    \includegraphics[width=.95\textwidth]{absorption}
    \caption{Зависимость от глубины плотности поглощенной энергии нерассеянных электронов в расчете на 1 электрон 1 для пучка радиусом 24.3 нм}
    \label{fig:absorption}
\end{figure}

%можно сделать вывод что , при пр

\newpage
\chapter*{Заключение}
\addcontentsline{toc}{chapter}{Заключение}
В результате использования низкого (около 20 кВ) напряжения у нас получается остров стабильности,
так как при использовании более низкого напряжения (\(\sim\)10 кВ) у нас получается что сильная диффузия электронов начинается прямо в слое пмма и сильно уменьшает качество маски при малых размерах ячеек.
при использовании высокого напряжения сильно проявляется эффект близости так же уменьшающий качество маски
в нашем же случае, по результатам расчета программы видно что доля обратно рассеянных электронов весьма мала, а поглощение начинается сразу же при вхождении в слой подложки в виду того что не велика энергия электронов, и влияние эффекта близости в таком случае весьма мало.


При изготовлении малых партий (большой номенклатуры) заказных логических схем из базового кристалла прямое рисование электронным лучем экономичнее, чем фотопечать через шаблон. Благодаря высокой разрешающей способности ЭЛ-литография будет и дальше использоваться при изготовлении шаблонов для световых, рентгеновских и ионных пучков. Кроме того, точность совмещения на каждом кристалле при ЭЛ-экспонировании составляет ±0.1 мкм, что является решающим преимуществом перед всеми остальными видами экспонирования.

\newpage
\addcontentsline{toc}{chapter}{Список использованных источников}
\def\bibname{СПИСОК ИСПОЛЬЗОВАННЫХ ИСТОЧНИКОВ}
\begin{thebibliography}{99}
\newcommand{\tabfix}{\hspace*{1.25cm}}
\bibitem{zheng-ming} \tabfix Zheng-ming,~L. Improved bipartitition model of electron
transport. II. Application to inhomogeneoue media / L.~Zheng-ming // Phys. Rev. B:
Condens. Matter. -- 1985. Vol.~32, N~2. -- P.~824--836.

\bibitem{smolar} \tabfix Smolar,~V. Backscattering and transmission for normal
incidence of electrons in small-angle and transport approximation /
V.~Smolar // Vacuum. 1994.~-- Vol.~45, N~6.~-- P.~609--615.

\bibitem{mikheev} \tabfix Михеев,~Н.~П. Транспортно-диффузионное приближение в теории
переноса электронов / Н.~П.~Михеев, В.~А.~Смоляр.~// Украинский физический
журнал.~-- 1985.~-- Т.~30, вып.~1.~-- С.~140--143.

\bibitem{smolar1} \tabfix Смоляр,~В.~А. Транспортно-малоугловое приближение в теории
переноса электронов средних энергий / В.~А.~Смоляр.~// Украинский физический
журнал.~-- 1988.~-- Т.~33, вып.~7.~-- С.~1072--1077.

\bibitem{smolar2} \tabfix Smolar,~V. Electron backscattering and penetration in the
small-angle and transport approximation model / V.~Smolar.~// Vacuum.~-- 1990.~--
Vol.~41, N~7--9.~-- P.~1718.--1720.

\bibitem{keiz} \tabfix Кейз,~К. Линейная теория переноса / К.~Кейз, П.~Цвайфель //
М.:~Мир, 1972 // 384~c.

\bibitem{korn} \tabfix Корн,~Г. Справочник по математике для научных работников и
инженеров /  Г.~Корн, Т.~Корн // М.:~Наука, 1968 // 720~c.

%\bibitem{smolar3} \tabfix Смоляр,~В.~А. / В.~А.~Смоляр.~// Украинский физический журнал.
%1982. Т.~27, вып.~10.~-- С.~1537. % не нашел статью

%литра с дипломника

\bibitem{1} \tabfix Reimer,~L. Scanning Electron Microscopy: Physics of Image Formation
and Microanalysis. / L.~Reimer.~-- N.~Y.~:~Springer, 2010 // 527~p.

\bibitem{2} \tabfix Практическая растровая электронная микроскопия / под ред.
Дж.~Гоулдстейна, Х.~Яковица // М.:~Мир, 1978 // 656~с.

\bibitem{3} \tabfix Растровый электронный микроскоп : учеб // метод. пособие /
А.~В.~Заболоцкий А.С. [и др.] // М.:~Изд-во~МФТИ, 2007 //
52~с. %Батурин, Е.А. Тишин, А.А. ЧуприкЪ

\bibitem{4} \tabfix Kyser,~D.~F. Monte Carlo simulation of spatially
distributed beams in electron-beam lithography / D.~F.~Kyser,
N.~S.~Viswanathan.~// Journal of Vacuum Science \& Technology B. 1975 V.~12,
N~6.~-- P.~1305--1308.

\bibitem{5} \tabfix Mankewich,~P.~M. Measurements of electron range and scattering in
high voltage e-beam lithography / P.~M.~Mankewich, L.~D.~Jackel,
R.~E.~Howard.~// Journal of Vacuum Science \& Technology B. 1985. V.~3, N~3.
P.~174--176.

\bibitem{6} \tabfix An image processing approach to fast, efficient proximity correction for
electron beam lithography / D.~Chow, [и др.] ~// Journal of Vacuum Science \& Technology B. 1983.
V.~1, N~4.~-- P.~1383--1390.%J.~McDonald, D.~King, W.~Smith,
%K.~Molnair, A.~Steckl.

\bibitem{7} \tabfix Accuracy of proximity correction in electron lithography after
development / V.~V.~Aristov, [и др.]~// Journal of Vacuum Science \& Technology B. V.~10,
N~6.~-- P.~2459--2467.
%B.~N.~Gaifullin, A.~A.~Svintsov, S.~I.~Zaitsev,
%R.~R.~Jede, H.~F.~Raith.
\bibitem{8} \tabfix Aristov,~V.~V. Guaranteed accuracy of the method of <<simple>>
compensation in electron lithography / V.~V.~Aristov, A.~A.~Svintsov,
S.~I.~Zaitsev.~// Microelectronic Engineering. 1990. V.~11, Is.s~1--4.
P.~641--644.

\bibitem{9} \tabfix Evaluation, verification and error determination of proximity
parameters (alpha, beta and eta) in electron beam lithography / S.~V.~Dubonos [и~др.]~// Microelectronic
Engineering. 1993. V.~21.~-- P.~293--296.%,
%B.~N.~Gaifullin, H.~F.~Raith, A.~A.~Svintsov, S.~I.~Zaitsev.

\bibitem{10} \tabfix Друкарёв,~Г.~Ф. Столкновение электронов с атомами и молекулами /
Г.~Ф.~Друкарёв // М.:~Наука, 1978 // 255~с.

\bibitem{11} \tabfix Joy,~D.~C. Monte Carlo Modeling for Electron Microscopy and
Microanalysis / D.~C.~Joy // Oxford University Press, 1995 // 224~p.

\bibitem{12} \tabfix Бёте,~Г. Квантовая механика / Г.~Бёте // М.:~Мир, 1965 // 333~с.

\bibitem{13} \tabfix Ландау,~Л.~Д. Курс теоретической физики : учеб. пособ. для вузов.
В 10 т. Том III. Квантовая механика (нерелятивистская теория) / Л.~Д.~Ландау, Е.~М.~Лившиц.~--
6-е~изд.,~испр. М.:~Физматлит, 2004 // 800~с.

\bibitem{14} \tabfix Deckert,~C.~A. Optimization of thin film wetting and adhesion
behavior / C.~A.~Deckert, D.~A.~Peters.~// Thin solid films. 1980. Vol.~68,
Is.~2.~-- P.~417--420.

\bibitem{15} \tabfix Моро,~У. Микролитография. Принципы, методы, материалы. В 2 т. / У.~Моро // М.:~Мир, 1990 // 1240~с.

\bibitem{16} \tabfix Blais,~P.~D. Edge acuity and resolution in positive type photo-.
resist systems / P.~D.~Blais~ // Solid-state Technologies. 1977. Vol.~20.~--
P.~76--79.

\bibitem{17} \tabfix Frish,~H. Sorption and transport in glassy polymers-a review /
H.~Frish.~// Polymer Engineering \& Science. 1980. Vol.~20, Is.~1.~-- P.~2--13.

\bibitem{18} \tabfix Chen,~S. Fickian diffusion of alkanes through glassy polymers:
Effects of temperature, diffusant size, and polymer structure / S.~Chen,
J.~Edin.~// Polymer Engineering \& Science. 1980. V.~20, Is.~1.~-- P.~40--50.

\bibitem{19} \tabfix Park,~G. Diffusion in Polymers / G.~Park; edited by J. Crank.~--
N.~Y.:~Academic Press, 1968.~-- 162p.

\bibitem{20} \tabfix Thomas,~L. A theory of case II diffusion / L.~Thomas,
J.~Windle.~// Polymer.~-- 1982.~-- Vol.~23, Is.~4.~-- P.~529--542.

\bibitem{21} \tabfix Ueberreiter,~K. Diffusion in Polymers / K.~Ueberreiter; edited by
J.~Crank, G.~Park.~-- N.~Y.~:~Academic Press, 1968.~-- 257p.

\bibitem{22} \tabfix Ueberreiter,~K. Velocity of dissolution of polystyrene /
K.~Ueberreiter, F.~Asmussen~// Journal of Polymer Science. 1957. Vol.~23,
Is.~103.~-- 81p.

\bibitem{23} \tabfix Handbook of Microlithography, Micromachining and Microfabrication
/ edited by P.~Rai-Choudhury // Washington:~SPIE Press, 1997 // 768~p.

\end{thebibliography}

\end{document}
