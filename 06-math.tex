\chapter{Математическая модель}
Задачи о транспорте быстрых электронов в многослойных мишенях в настоящее
время решаются почти исключительно методом Монте-Карло, феноменологические
модели здесь практически непригодны, а аналитические методы решения кинетического
уравнения недостаточно разработаны даже для однородных слоев и ограничиваются
частными задачами — это либо задача об обратном рассеянии, либо, что значительно
реже, задача об энерговыделении для широкого пучка, падающего на однородную
мишень. Известным нам исключением является работа [1], в которой двухгрупповая
модель расщепления кинетического уравнения Спенсера была применена для расчета
энерговыделения широкого пучка в двухслойной мишени. Однако в этой работе
уравнение Спенсера записано для плотности потока электронов в фазовом пространстве,
включающем остаточный пробег электронов, и поэтому оно неприменимо для
неоднородных сред, так как в таких средах нет однозначного соответствия между
энергией электронов и их остаточным пробегом и пробег не имеет физического смысла.
Для многослойных мишеней следует применять кинетическое уравнение для электронов в
фазовом пространстве. Включающем энергию, как это сделано в работе [2].
В наших предыдущих работах [3–5] было описано расщепление транспортного
уравнения переноса электронов на основе соединения в единой математической модели
двух предельных приближений — малоуглового и транспортного. В этой модели частота
упругих столкновений аппроксимируется суммой изотропного и малоуглового
компонентов, а плотность потока электронов представляется в виде суммы двух
компонентов — сохраняющего первоначальное направление и диффундирующего.
В случае нормального падения на образец, состоящий из n слоев, точечного пучка
электронов с начальной энергией $E_0$ не происходит слишком большого усложнения
задачи, и она остается в пределах возможностей применяемых аналитических и
численных методов. Плотность потока электронов в среде также представляется в виде
суммы
(1)
\begin{equation}
N(z,\vec{r},\vec{\Omega},E)=N_f (z,\vec{r},\vec{\Omega},E)+N_d (z,\vec{r},\vec{\Omega},E),
\label{eq:1}
\end{equation}
где $N_f$ и $N_d$ – плотности потока электронов, сохраняющих направление и
диффундирующих, ось $z$ с началом в точке падения пучка направлена в глубину мишени,
$\vec{r}$ – вектор, ортогональный оси $z$, $\vec{\Omega}$ – направление скорости и $E$ – энергия электронов.
Для функции $N_f$ было получено аналитическое решение в однородном слое [5] и в
многослойной структуре [2]. Здесь мы запишем аналитическое решение для $N_f$ в
многослойной мишени в несколько иной форме. Транспортная частота и средние потери
энергии в такой мишени являются ступенчатыми функциями относительно переменной $z$
и могут быть представлены с помощью характеристических функций $\chi_i(E)$ i-того слоя –
произведения единичных ступенчатых функций Хевисайда $U(x)$


\begin{equation}
\bar{\varepsilon}(E,z)=\sum_{i=1}^n \chi_i (E)\bar{\varepsilon}^{(i)} (E),
\label{eq:2}
\end{equation}

\begin{equation}
W_tr(E,z)=\sum_{i=1}^n \chi_i (E)W_{tr}^{(i)} (E),
\label{eq:3}
\end{equation}

\begin{equation}
W_{sa,tr}(E,z)=\sum_{i=1}^n \chi_i (E)W_{sa,tr}^{(i)} (E),
\label{eq:4}
\end{equation}
где
$ \chi_i(E)=U(E_{i-1}-E)U(E-E_i) $,
$Ei$ – энергия прямоидущих электронов на правой границе i-того слоя, $W_tr(E,z)$ –
транспортная частота столкновений, $\bar{\varepsilon}(E,z)$ – потеря энергии на единицу пробега, $W_{sa,tr}(E,z)$ – малоугловой компонент частоты столкновений.
Плотность малоуглового компонента потока имеет вид [2]

\begin{equation}
N_f(z,\vec{r},\vec{\Theta},E)=\frac{\delta[z-s(E,z)]}{[\pi \bar{\varepsilon}(E,z)\sigma^2(z)]}\times \exp \frac{[\frac{(-\vec{\Theta}^2 C_2(z)-\vec{r}^2C_0 (z)+2(\vec{r}\vec{\Theta})C_1(z))}{\sigma^2(z)}]}{[\pi\sigma^2(z)]} \times \exp\left(-\int_{E}^{E_0} \frac{w_{tr}(E',z)}{\bar{\varepsilon}(E',z)}dE'\right),
\label{eq:6}
\end{equation}

где

\begin{equation}
C_k(z)=2 \int_{E}{E_0} \frac{w_{sa,tr}(E',z)[z-s(E',z)]^k}{\bar{\varepsilon}(E',z)} dE',\>\> к=0,1,2.,
\label{eq:7}
\end{equation}

\begin{equation}
\sigma^2(z)=C_0(z)C_2(z)-C_1^2(z),
\label{eq:8}
\end{equation}

\begin{equation}
s(E,z)=\sum_{i=1}^{n} \int_{E}^{E_0} \frac{\chi_i}{\bar{\varepsilon}(E',z)} dE',
\label{eq:9}
\end{equation}

Функция $s(E, z)$ имеет смысл среднего пути электрона, сохраняющего первоначальное
направление вдоль оси $z$, когда его энергия уменьшается от значения $E_0$ в точке $z=0$ до $E$
на глубине $z$.
Плотность потока диффузного компонента $N_d(z,\vec{r},\vec{\Omega},E)$ является решением
уравнения [2]

\begin{gather}
-\frac{\partial}{\partial E}[\bar{\varepsilon}(E,z)N_d(z,\vec{r},\vec{\Omega},E)]+\vec{\Omega} \bigtriangledown N_d (z,\vec{r},\vec{\Omega},E)= \nonumber \\
\frac{1}{4\pi} \int d\vec{\Omega}'W_{tr}(E,z)[N_d(z,\vec{\Omega}',E)-N_d(z,\vec{r},\vec{\omega},E)]+\frac{1}{4\pi}w_{tr}(E,z) \int d\vec{\Omega}'N_f(z,\vec{r}',\vec{\Omega}',E),
\label{eq:10}
\end{gather}


Используя аналитическое решение (6) и численное решение уравнения (10), мы
можем определить любые дифференциальные и интегральные характеристики переноса
электронов в мишени. Расчеты распределений обратно рассеянных и прошедших
электронов обсуждались ранее [2]. В настоящей работе мы рассмотрим распределение
плотности поглощенной энергии электронов.
Учитывая аксиальную симметрию задачи, представим плотность поглощенной
энергии в виде

\begin{equation}
W(z,r)=\int_{0}^{E_0} dE\bar{\varepsilon}(E,z) \int d\vec{\Omega}E(z,\vec{r},\vec{\Omega},E)=W_f(z,r)+W_d(z,r),
\label{eq:11}
\end{equation}
где

\begin{equation}
W(z,r)=\int_{0}^{E_0} dE\bar{\varepsilon}(E,z)2\pi \int_{-1}{1} d(\cos \Theta)N_f(z,r,\Theta ,E),
\label{eq:12}
\end{equation}

\begin{equation}
W_{d}(z,r)=\int_{0}^{E_0} dE\bar{\epsilon}(E,z)2\pi \int_{-1}^{1} d(\cos \Theta)N_d(z,r,\Theta ,E)=\int_{0}^{E_0}dE\vec{\varepsilon}(E,z)N_{d0}(z,r,E),
\label{eq:13}
\end{equation}

\begin{equation}
N_{do}(z,r,E)=2\pi \int_{-1}{1} d(\cos \Theta)N_d(z,r,\Theta,E),
\label{eq:14}
\end{equation}

Здесь использована цилиндрическая система координат,$r=|\vec{r}|$ -- радиальное
расстояние, $\Theta$ — угол между осью $z$ и вектором $\vec{\Omega}$ .
Используя (6) и выполняя интегрирование в (12), получаем гауссиан
\begin{equation}
W_f(z,r)=\frac{A_f(z)}{\pi \sigma_f^2(z)} \exp \left(\frac{r^2}{\sigma_f^2(z)}\right),
\label{eq:15}
\end{equation}

с полушириной $\sigma_f(z)=\sqrt{C_2 (z)}$ и амплитудой
\begin{equation}
A_f(z) = \int\limits_0^{E_0}\delta( z − s(E, z))\exp\left(\int\limits_0^{E_0}\frac{w_{tr} ( E′, z)}{\bar{\varepsilon}(E′, z)}dE′\right) dE
\end{equation}
Используя представление двумерной \(\delta\)-функции с помощью последовательности непрерывных функций [7]
\begin{equation}
\delta(\vec{r}) = \lim_{\sigma\rightarrow0} \frac{\exp\left(−\frac{r^2}{\sigma^2}\right)}{\pi\sigma^2}
\end{equation}
из (16) получаем
\begin{equation}
\lim_{z\rightarrow0} W_f(z, r) = A_f(0)\delta(\vec{r}),
\end{equation}
так что распределение \( W_f(z, r) \) является сингулярным вблизи точки падения
электронного пучка. Отметим, что
\begin{equation}
    A_f(0) = \bar{\varepsilon}(E=0, z=0).    
\end{equation}
Плотность поглощенной энергии диффузного компонента \(W_d(z, r)\) может быть
рассчитана с помощью функции \(N_{d0}(z, r, E),\) которая является решением краевой задачи.
Введя для удобства новую переменную \(\tau=E_0-E\) -- остаточную энергию электронов,
представим задачу для функции \(N_{d0}(z, r, \tau)\) в стандартном виде [12]
\begin{align}
& \frac{\partial N_{d0}}{\partial\tau} = \frac{\partial}{\partial z}\left[A (\tau , z )\frac{\partial N_{d0}}{\partial z}\right]
+ \frac{1}{r}\frac{\partial}{\partial r}\left[rA(\tau, z)
\frac{\partial N_{d0}}{\partial r}\right] + B (\tau , z )
\frac{\partial N_{d0}}{\partial z} − G (\tau , z )N_{d0}
+ \\
& \exp\left( −\frac{r^2}{C_2(z)}\right)\pi C_2(z)\frac{g(\tau, z)}{\bar{\varepsilon} (\tau , z ) }\delta ( z − s(\tau , z )).
\nonumber
\end{align}
где
\begin{align}
& A (\tau, z) = \frac{1}{\bar{\varepsilon}(\tau, z)w_{tr} (\tau, z)}, \\
& B (\tau, z) = \frac{1}{w_{tr}(\tau, z)}\frac{\partial\frac{1}{\bar{\varepsilon}(\tau, z)}}{\partial z}, \\
& G (\tau, z) = \frac{\partial\ln\bar{\varepsilon}(\tau, z )}{\partial\tau},\\
& g(\tau, z) = \frac{w_{tr}(\tau , z )}{\bar{\varepsilon} (\tau , z )}\exp\int( −\int_0^{E_0} \frac{w_{tr}(\tau′, z )}{\bar{\varepsilon} (\tau′, z)}d\tau′.    
\end{align}
Для функций \( \tau (\tau , z ), w_{tr}(\tau , z)\) и \(s(\tau,z)\) используем аппроксимации, описанные в [8].
Пусть \(\xi_1,\ldots,\xi_n\) -– точки разрыва функции \(\bar{\varepsilon}(\tau , z)\) по отношению к переменной \( z \) (эти точки являются границами слоев). Тогда
\begin{equation}
B(\tau, z) = −\frac{1}{w_{tr}(\tau, z)}\sum_{i=1}^{n} \bar{\varepsilon} −1 (\tau ,\xi_i ) \delta ( z − \xi i ) ,    
\end{equation}
где \(\bar{\varepsilon}^{−1}\) обозначает скачек
\begin{equation}
    \bar{\varepsilon}^{−1} (\tau ,\xi_i ) = \bar{\varepsilon}^{−1} (\tau,\xi_i + 0 ) − \bar{\varepsilon}^{−1} (\tau ,\xi_i − 0 ).
\end{equation}

В однослойной мишени имеем \( B(\tau,z)=0 \).
Уравнение (20) должно быть решено в области
\begin{equation}
    (z, r,\tau) \in [0, z_\text{макс}]\times [0, r_\text{макс} ]\times[0, E_0],    
\end{equation}
где величина \( z_\text{макс} \) определяется структурой мишени, а значение \( r_\text{макс} \) должно быть
конечным из вычислительных соображений. Представляется естественным выбрать \( r_\text{макс} \)
много большим, чем максимальная оценка длины свободного пробега в слоях.
Граничные и условия Маршака в стандартной форме [6] имеют вид
\begin{align}
& \left[0,5 \frac{N_{d0}}{\bar{\varepsilon}(\tau, z)} − A (\tau , z)\frac{\partial N_{d0}}{\partial z}\right]_{z=0} = 0 ,\\
& \left[0,5 \frac{N_{d0}}{\bar{\varepsilon}(\tau, z)} − A (\tau , z)\frac{\partial N_{d0}}{\partial z}\right]_{z=z_\text{макс}} = 0 ,\\
& \left[0,5 \frac{N_{d0}}{\bar{\varepsilon}(\tau, z)} − A (\tau , z)\frac{\partial N_{d0}}{\partial r}\right]_{r=r_\text{макс}} = 0,\\
& \left[\frac{\partial N_{d0}}{\partial r}\right]_{r=0} = 0\\
& N_{d0}(z, r, \tau=0 ) = 0.
\end{align}
