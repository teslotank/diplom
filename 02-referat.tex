\begin{center}
РЕФЕРАТ
\end{center}

Целью данной работы является изучение явления эффекта близости в электронной литографии низких энергий.
Его проявление в случае облучения электронным потоком слоя резиста ПММА на кремниевой подложке. В процессе работы были проведены расчеты по определение параметров поглощения энергии, обратного рассеяния, и диффузии на различных глубинах и при различной плотности энергии пучка.
\vspace*{1cm}

\hspace{-1.25cm}Ключевые слова: электронная литография, ПММА, метод Монте-Карло, эффект близости, диффузионные электроны, резист, обратно рассеянные электроны.
 \vspace*{1cm}

\begin{center}
ABSTRACT
\end{center}

The purpose of this operation is the study of the phenomenon of effect of closeness in electronic lithograph of low energies.
Its manifestation in case of radiation by an electronic flow of a layer of PMMA resist on the Silicon substrate. In the course of operation calculations on determination of parameters of a poglavshcheniye of energy, back scattering, and diffusion at rvzlichny depths were carried out and in case of different density of energy of a bundle.
\vspace*{1cm}

\hspace{-1.25cm}Key words: electronic lithograph, PMMA, Monte-Carlo method, effect of closeness, diffusion electrons, resist, back scattered electrons.
