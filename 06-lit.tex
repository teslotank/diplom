порядок можно поменять, нумирация херовая, ну короче все плохо, печаль бяда и безысходность(
литра с методы
1. Zheng-ming L. // Phys. Rev. B: Condens. Matter. 1985. V.32. No2. P. 824.
2. Smolar V.// Vacuum. 1994. V. 45. No6. P. 609.
3. Михеев Н. П., Смоляр В.А.// УФЖ. 1985. Т.30. No1. С. 140.
4. Смоляр В.А. // УФЖ. 1988. Т. 33. No 7. С. 1072.
5. Smolar V.// Vacuum. 1990. V. 41. No7–9. P. 1718.
6. Кейз К., Цвайфель П. Линейная теория переноса. М.: Мир,. 1972.
7. Корн Г., Корн Т. Справочник по математике для научных работников и инженеров. М.:
8. Наука, 1968.
9. СмолярВ.А.// УФЖ. 1982. Т. 27. No 10. С. 1537.
литра с дипломника
1. Reimer. SEM. Physics of Image Formation and Microanalysis
2. "Практическая растровая электронная микроскопия", под ред.
Гоулдстейна Дж. и Яковца Х. М.: Мир, 1978
3.
А.В. Заблоцкий, А.С. Батурин, Е.А. Тишин, А.А. Чуприк.
"Растровый электронный микроскоп: Лабораторная работа". – М.:
МФТИ, 2007. – 52 с.
4.
D. F. Kyser and N. S. Viswanathan. "Monte Carlo simulation of spatially
distributed beams in electron-beam lithography," Journal of Vacuum
Science & Technology B, vol. 12 No6, p. 1305-1308 (1975).
5.
P. M. Mankewich, L. D. Jackel, and R. E. Howard. “Measurements of
electron range and scattering in high voltage e-beam lithography”,
Journal of Vacuum Science & Technology B, vol. 3 No3, p. 174-176
(1985).
6.
D. Chow, J. McDonald, D. King, W. Smith, K. Molnair, and A. Steckl.
”An image processing approach to fast, efficient proximity correction for
electron beam lithography”, Journal of Vacuum Science & Technology
B, vol. 1 No4, p. 1383-1390 (1983).
7.
V. V. Aristov, B. N. Gaifullin, A. A. Svintsov, S. I. Zaitsev, R. R. Jede
and H.F. Raith. “Accuracy of proximity correction in electron
lithography after development”, Journal of Vacuum Science &
Technology B, vol. 10 No 6, p. 2459-2467 (1992).
8.
V. V. Aristov, A. A. Svintsov and S. I. Zaitsev. “Guaranteed accuracy of
the method of ‘simple’ compensation in electron lithography”,
Microelectronic Engineering, vol. 11 Issues 1-4, p. 641-644 (1990).
9.
S.V. Dubonos, B.N. Gaifulin, H.F. Raith, A.A. Svintsov, and S.I. Zaitsev,
Microelectron. Eng. 21, 293 (1993)
10. (3-1) Друкарёв Г.Ф. «Столкновение электронов с атомами и
молекулами» 1978 М. «Наука»
39
11. (3-18)D. C. Joy, “Monte Carlo Modeling for Electron Microscopy and
Microanalysis”, Oxford University Press, 1995.
12. (3-4) Бёте «Квантовая механика»
13. (3-5) Л.Д. Ландау Е.М. Лившиц «Теоретическия физика. Том III.
Квантовая механика. Нерелятивистская теория» 1989 М. Наука.
14. C. A. Deckert and D. A. Peters. “Optimization of thin film wetting and
adhesion behavior”, Thin solid films, vol. 68 Issue 2, p. 417-420 (1980).
15. У. Моро, “Микролитография. Принципы, методы, материалы”//
«Мир», (1990).
16. P. D. Blais. “Edge acuity and resolution in positive type photo-. resist
systems”// Solid-state Technol., vol. 20, p. 76-79 (1977).
17. H. Frish. “Sorption and transport in glassy polymers-a review”// Polym.
Eng. Sci., vol. 20 Issue 1, p. 2-13 (1980).
18. S. Chen and J. Edin. “Fickian diffusion of alkanes through glassy
polymers: Effects of temperature, diffusant size, and polymer structure”//
Polym. Eng. Sci., vol. 20, p. 40-50 (1980).
19. G. Park. “Diffusion in Polymers”// edited by J. Crank and G. Park,
Academic Press, New York, Chapter 5, p. 140-162 (1968).
20. L. Thomas and J. Windle. “A theory of case II diffusion”// Polymer, vol.
23 Issue 4, p. 529-542 (1982).
21. K. Ueberreiter. “Diffusion in Polymers”// edited by J. Crank and G. Park,
“Academic Press”, New York, , Chapter 5, p. 219-257 (1968).
22. K. Ueberreiter and
F.
Asmussen. “Velocity of dissolution of
polystyrene”// J. Pol. Sci., vol. 23 Issue 103 , p. 75-81 (1957).
23. “Handbook
of
Microlithography,
Micromachining
and
Microfabrication”// edited by P. Rai-Choudhury, SPIE, Chapter 2,
(1997).


