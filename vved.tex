
\begin{titlepage}

\chapter*{Введение}
\addcontentsline{toc}{chapter}{Введение}

Миниатюризация элементов интегральных схем есть способ увеличения их
производительности и эффективности. Поэтому существует потребность как
в развитии наноструктурирования, так и создания специализированных
структур на чипах с элементами, имеющими нанометровые размеры в таких
областях как: оптоэлектроника, рентгеновская оптика, исследования в
области физики низких температур и квантово-размерных эффектов, новых
материалов, таких как двумерный графен и т.д.
Структурирование с помощью электронной литографии является самым
удобным методом создания объектов ввиду своей гибкости и оперативности.


\end{titlepage}
 
